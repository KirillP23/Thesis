{\singlespacing
   \begin{flushright}
      Kirill Paramonov \\
      June 2018 \\
      Mathematics \\
   \end{flushright}
}

\bigskip

\begin{center}
   Essays in Combinatorics: Crystal on Shifted Primed Tableaux, bigraded Fibonacci Numbers and Data Mining on Social Graphs \\
\end{center}

\section*{Abstract}

This dissertation consists of three research projects in combinatorics, which come from three distinct branches: algebraic combinatorics, enumerative combinatorics and application of combinatorics to graph data mining.

The first project explores the Schur decomposition of type $C$ Stanley symmetric functions via crystal structure on shifted primed tableaux.
We present a connection between unimodal factorizations that form type $C$ Stanley symmetric functions, and shifted primed tableaux using Kra\'skiewicz insertion.
Then we present crystal operators on the set of shifted primed tableaux, which in turn allows us to find the Schur decomposition of its characteristic function (also known as $P$-Schur function).
The class of shifted primed tableaux together with its crystal structure has been implemented in Sage.

In the second project we explore combinatorial properties of simultaneous $(a,b)$-cores with distinct parts.
We use the bijections between $(a,b)$-cores and abacus diagrams to give another proof that the number of $(a,a+1)$-cores with distinct parts is the Fibonacci number $F_{a+1}$.
We also provide a proof for the number of maximal-sized $(a,a+1)$-cores with distinct parts, and for the average size of such cores.
Using bijection between cores and Dyck paths, we show that the number of $(2k-1, 2k+1)$-cores with distinct parts is equal to $2^{2k-2}$.
We then further generalize our ideas to $(a,as+1)$-cores with distinct parts and introduce the $q,t$-Fibonacci numbers $F_a^{(s)}(q,t)$, motivated by $area$ and $bounce$ statistics of Dyck paths that generate $q,t$-Catalan numbers.
Bigraded Fibonacci numbers have nice recursive relations that turn out to be equivalent to Andrews $q$-equations related to Rogers-Ramanujan identities.

In the third project, we develop new method of estimating the number of appearances of a specific motif (like wedge or triangle) in a graph.
The method uses Horowitz-Thompson inverse probability weighting for the subgraphs sampled in a procedure called lifting.
We introduce three modifications of the lifting estimator: ordered, unordered and shotgun estimators.
Our method is universal and has better mixing time and variance compared to other methods, which has been demonstrated both theoretically and experimentally.
The method serves as a starting point of further development of machine learning algorithms on graphs.







