
This thesis consists of three parts, each corresponding to a single research project.
Although the connection between the projects is loose, each of them represents a different part of Combinatorics: Chapter~\ref{ch:crystals} for Algebraic Combinatorics, Chapter~\ref{ch:fibonacci} for Enumerative Combinatorics, and Chapter~\ref{ch:lift} for applications of Combinatorics in Data Mining algorithms.
Therefore, the reader can follow the chapters in any order, depending on their academic interests.

In this chapter we introduce the necessary background and describe the main results of each chapter to follow.

\section{Crystals and type $C$ Stanley Symmetric Functions.}

The goal of this project is to find a Schur decomposition of the type $C$ Stanley symmetric functions.
We'll start with general theory of symmetric functions.

\subsection{Schur functions}

A \defn{Young diagram} is a finite collection of boxes, or cells, arranged in left-justified rows, with row lengths in non-decreasing order.
The non-decreasing row lengths form a \defn{partition} $\lambda=(\lambda_1,\ldots,\lambda_s)$, called a \defn{shape} of the Young diagram.

A \defn{semistandard Young tableau} $T$ is obtained by filling the boxes of the Young diagram with positive integers so that entries weakly increase along each row, and strictly increasing along each column.
The \defn{weight} of a semistandard Young tableau $w(T)$ is vector $(w_1, w_2, \ldots)$, where $w_i$ counts the number of occurrences of the number $i$ in $T$.

A \defn{Schur function} $s_\lambda(\mathbf{x})$ is defined to be the characteristic function of the set of all semistandard Young tableaux of shape $\lambda$.
That is,
\begeqno
	s_\lambda(x_1,x_2,\ldots) = \sum_T x_1^{w_1}x_2^{w_2}\ldots = \sum_T \mathbf{x}^{w(T)},
\eneqno
where the sum is taken over all semistandard Young tableau of shape $\lambda$.
Here, we used notation $\mathbf{x} = (x_1,x_2,\ldots)$ to represent the infinite vector of variables $x_i$.

Schur functions play an important role in algebraic combinatorics and representation theory because they represent characters of irreducible representations of the symmetric group, so they often serve as building blocks for characters of symmetric group representations.
Therefore, decomposing a symmetric function $F$ into a linear combination of Schur functions is important for understanding representation corresponding to function $F$.

The representation $F(\mathbf{x}) = \sum_\lambda K_\lambda s_\lambda(\mathbf{x})$ is called a \defn{Schur decomposition}.
Our goal is to find a Schur decomposition of type $C$ Stanley symmetric functions.

\subsection{Stanley symmetric functions}

The \defn{Coxeter group of type $A_n$}, denoted by $S_n$, is a finite group generated by $\{s_0, \ldots, s_{n-1}\}$ subject to the quadratic relations
$s_i^2 = 1$ for all $i \in I = \{0,\ldots,n-1\}$, the commutation relations $s_i s_j = s_j s_i$ provided $|i-j|>1$, and the
braid relations $s_i s_{i+1} s_i = s_{i+1} s_i s_{i+1}$ for all $i$.

The \defn{Coxeter group of type $C_n$}, denoted by $T_n$, has the same generators and relations as $S_n$, except the braid relation $s_0 s_1 s_0 = s_1 s_0 s_1$ from type $A$ changes to $s_0 s_1 s_0 s_1 = s_1 s_0 s_1 s_0$ in type $C$.

It is often convenient to write down an element of a Coxeter group as a sequence of indices of $s_i$ in the product representation of the element. 
For example, the element $w\in T_3$ with $w = s_2 s_1 s_2 s_1 s_0 s_1 s_0 s_1$ is represented by the word ${\bf w} = 2120101$. 
A word of shortest length $\ell$ is referred to as a \defn{reduced word} and $\ell(w):=\ell$ is referred as the length of $w$. 
The set of all reduced words of the element $w$ is denoted by $R(w)$.

\begin{example}
The set of reduced words for type $C$ element $w = s_2 s_1 s_2 s_0 s_1 s_0$ 
is given by
$$R(w) = \{ 210210, 212010, 121010, 120101, 102101 \}.$$
\end{example}

An element $v\in S_n$ is called \defn{decreasing} if there is a reduced word $i_1\cdots i_m$ for $v$ such that $i_1> \cdots > i_m$.
The identity is considered to be decreasing.
Given $w\in S_n$, a \defn{decreasing factorization} of $w$ is a factorization of $w = w^k\cdots w^1$ with $\ell(w) = \ell(w^k)+ \cdots + \ell(w^1)$ and each $w^i$ is decreasing.

We denote the set of all decreasing factorizations of $w\in S_n$ by $D(w)$.
Define the \defn{weight} of the decreasing factorization  $w^k \cdots w^1$ to be the vector $\wt(w^k \cdots w^1) = \left(\ell(w^1),\ldots,\ell(w^k)\right)$
The Stanley symmetric function of type $A$ is defined as
\begeqno
	F^A_w(${\bf x}$) = \sum_{w^k\cdots w^1\in D(w)} x_1^{\ell(w^1)}\cdots x_k^{\ell(w^k)}.
\eneqno

We define similar notions for type $C$.
We say that a reduced word $a_1 a_2 \ldots a_\ell$ is \defn{unimodal} if there exists an index $v$, such that 
$$a_1 > a_2 > \cdots > a_v < a_{v+1} < \cdots < a_\ell.$$

Consider a reduced word $\textbf{a} = a_1 a_2 \ldots a_{\ell(w)}$ of a Coxeter group element $w$. 
A \defn{unimodal factorization} of $\textbf{a}$ is a factorization 
$\mathbf{A} = (a_1 \ldots a_{\ell_1}) (a_{\ell_1+1} \ldots a_{\ell_2}) \cdots (a_{\ell_r + 1} \ldots a_L)$ such that each factor 
$(a_{\ell_i+1} \ldots a_{\ell_{i+1}})$ is unimodal. Factors can be empty.

For a fixed Coxeter group element $w$, consider all reduced words $R(w)$, and denote the set of all unimodal 
factorizations for reduced words in $R(w)$ as $U(w)$. 
Given a factorization $\mathbf{A} \in U(w)$, define the \defn{weight} of a factorization $\wt(\mathbf{A})$ to be the vector 
consisting of the number of elements in each factor. Denote by $\nz(\mathbf{A})$ the number of non-empty factors of 
$\mathbf{A}$. 

\begin{example}
For the factorization $\mathbf{A} = (2102)()(10) \in U(s_2 s_1 s_2 s_0 s_1 s_0)$, we have $\wt(\mathbf{A}) = (4,0,2)$ 
and $\nz(\mathbf{A}) = 2$.
\end{example}





















