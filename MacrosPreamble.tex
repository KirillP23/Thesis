% Use this file to load additional packages or define macro
% commands that will be used in the dissertation.
%
\makeatletter    
\def\l@subsection{\@tocline{1}{0,2pt}{2pc}{8mm}{\ \ }} 
\def\l@subsubsection{\@tocline{2}{0,2pt}{4pc}{8mm}{\ \ }}

\usepackage{listings}
\usepackage{color}
\usepackage[utf8]{inputenc}

\definecolor{dkgreen}{rgb}{0,0.6,0}
\definecolor{gray}{rgb}{0.1,0.1,0.1}
\definecolor{mauve}{rgb}{0.58,0,0.82}

\lstset{frame=tblr,
  language=Python,
  aboveskip=5mm,
  belowskip=5mm,
  showstringspaces=false,
  extendedchars=true,
  basicstyle=\linespread{1.1}\small\ttfamily,
  numbers=none,
  numberstyle=\tiny\color{gray},
  keywordstyle=\color{blue},
  commentstyle=\color{dkgreen},
  stringstyle=\color{black},
  tabsize=4
}

\usepackage{graphics, graphicx, xcolor}

\definecolor{darkred}{RGB}{139,0,0} % darkred color
\newcommand{\darkred}{\color{darkred}} % darkred command
\newcommand{\defn}[1]{\emph{\darkred #1}} % emphasis of a definition
\definecolor{darkblue}{RGB}{0,0,139} 
\newcommand{\darkblue}{\color{darkblue}} 

\usepackage{hyperref}
\hypersetup{colorlinks=true, citecolor=darkblue, linkcolor=darkblue, urlcolor=darkblue}
\usepackage{caption}
\usepackage{subcaption} %%% optional but probably necessary
\usepackage{amsfonts,amsmath,amsxtra,amsthm,amssymb,tikz,latexsym}
\usepackage{mathrsfs}

\usepackage{booktabs}
\usepackage{algorithm,algorithmic}
\usepackage{dsfont}
\usepackage{multirow}

\usepackage[vcentermath, enableskew]{youngtab}
\usepackage{lipsum,  verbatim}
\usepackage{mathtools}
\usepackage{txfonts}


\newcommand{\onep}{1'}
\newcommand{\twop}{2'}
\newcommand{\threep}{3'}
\newcommand{\fourp}{4'}
\newcommand{\fivep}{5'}
\newcommand{\sixp}{6'}
\newcommand{\sevenp}{7'}
\newcommand{\eightp}{8'}
\newcommand{\ninep}{9'}
\newcommand{\iprime}{i'}
\newcommand{\jprime}{j'}
\newcommand{\minustwop}{\textrm{-}2'}
\newcommand{\minusonep}{\textrm{-}1'}
\newcommand{\minustwo}{\textrm{-}2}
\newcommand{\minusone}{\textrm{-}1}
\newcommand{\boldtwo}{\mathbf{2}}
\newcommand{\boldthree}{\mathbf{3}}
\newcommand{\boldtwop}{\mathbf{2'}}
\newcommand{\boldthreep}{\mathbf{3'}}
\newcommand{\boldi}{\boldsymbol{i}}

\newcommand{\begeq}{\begin{equation}}
\newcommand{\eneq}{\end{equation}}
\newcommand{\begeqno}{\begin{equation*}}
\newcommand{\eneqno}{\end{equation*}}
\newcommand{\wt}{\mathrm{wt}}
\newcommand{\nz}{\mathrm{nz}}

\def\ind{\mathds{1}}
\def\E{\mathbb{E}}
\def\P{\mathbb{P}}
\def\Cov{\mathrm{Cov}}
\def\Var{\mathrm{Var}}
\def\deg{\mathrm{deg}}
\def\cV{\mathcal{V}}
\def\cN{\mathcal{N}}
\def\co{\mathrm{co}}

\newcommand\independent{\protect\mathpalette{\protect\independenT}{\perp}}
\def\independenT#1#2{\mathrel{\rlap{$#1#2$}\mkern2mu{#1#2}}}

%%% environments
%\newtheorem{theorem}{Theorem}[section]
%\newtheorem{proposition}{Proposition}[section]
%\newtheorem{lemma}{Lemma}[section]
%\newtheorem{corollary}{Corollary}[section]
%\newtheorem{remark}{Remark}[section]
%\theoremstyle{definition}
%\newtheorem{definition}[theorem]{Definition}
%\newtheorem{example}[theorem]{Example}

\newtheorem{theorem}{Theorem}[chapter]
\newtheorem{proposition}[theorem]{Proposition}
\newtheorem{lemma}[theorem]{Lemma}
\newtheorem{definition}[theorem]{Definition}
\newtheorem{corollary}[theorem]{Corollary}
\newtheorem{example}[theorem]{Example}
\newtheorem{remark}[theorem]{Remark}
\newtheorem{conjecture}{Conjecture}
\newtheorem{claim}{Claim}
