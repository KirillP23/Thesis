\graphicspath{ {ch3_figures/} }

This Chapter continues from the Section~\ref{intro:fibonacci}, where we introduced notions of $(a,b)$-cores, Dyck paths and abacus diagrams.
We start by providing results for $(a,as+1)$-cores with distinct parts and their description in terms of abacus diagrams.

The results of this Chapter were published in the journal of Discrete Mathematics~\cite{Fibonacci.2018}.

\section{Simultaneous cores with distinct parts.}
\label{section.description}
Let us consider an $a$-core $\lambda$. We can express the condition that $\lambda$ has distinct parts in terms of the boundary $\partial(\lambda)$.

\begin{proposition}
An $a$-core $\lambda$ has distinct parts if and only if for each positive SE-step $j$ in $\partial(\lambda)$, the steps $j-1$ and $j+1$ are NE-steps.
\end{proposition}

\begin{proof}
Suppose there are two consecutive SE-steps $j+1$ and $j$ in $\partial(\lambda)$. Then there are two rows $row_i$ and $row_{i+1}$ in the Young diagram of $\lambda$ that are bordered by steps $j+1$ and $j$ from NE-side. Then $\lambda_i = \mathrm{length}(row_i)$ and $\lambda_{i+1} = \mathrm{length}(row_{i+1})$ are equal, and thus we arrive to a contradiction.
\end{proof}

That property can be formulated in terms of vectors $d=\mathbf{abac}(\lambda)$. Given a subset $S$ of $\{0,1,\ldots, a-1\}$, we call $S$ an \defn{$a$-sparse set} if for any two elements $n,m \in S,\ |n-m| \neq 1$ and $0\notin S$. Whenever size $a$ of the superset is not relevant, we will refer to $S$ as just a \defn{sparse set}.

The \defn{support} of the vector $d$ (denoted by \defn{$\mathrm{supp}(d)$}) is defined to be the set of all indexes $i$ with $d_i > 0$. We call vectors $d = (d_{0}, \ldots, d_{a-1})$ with $a$-sparse support to be \defn{$a$-sparse vectors}.

\begin{proposition}
\label{prop.distinct}
An $a$-core $\lambda$ has distinct parts if and only if the vector $d=\mathbf{abac}(\lambda)$ is an $a$-sparse vector.
\end{proposition}

\begin{proof}
($\Rightarrow$) Consider an index $i \in \{1,\ldots, a-1\}$ such that $d_i > 0$. For the corresponding $a$-abacus diagam that means the bead on runner $i$ in position $0$ is a black one, and the $i$-th step of $\partial(\lambda)$ is an SE-step. If $\lambda$ has distinct parts, that means steps $i-1$ and $i+1$ of $\partial(\lambda)$ are NE-steps and thus runners $i-1$ and $i+1$ have no black beads on nonnegative positions, and $d_{i-1} = d_{i+1} = 0$. 

($\Leftarrow$) Conversely, suppose $d=\mathbf{abac}(\lambda)$ is an $a$-sparse vector. Any step SE-step $j$ of $\partial(\lambda)$ corresponds to some black bead on runner $i$ in position $k$. Because of the sparsity, beads on runners $i-1$ and $i+1$ in position $k$ must be white ones, and thus steps $j-1$ and $j+1$ of $\partial(\lambda)$ must be NE-steps.
Note that for any SE-step $j$ corresponding to the runner $i=1$, step to the left of $j$ is on the runner 0 and thus is always an NE-step. Similarly, when $i=a-1$ the step to the right of $j$ is on the runner 0 in positive position and thus is always an NE-step.
\end{proof}

We now consider an additional structure of simultaneous $(a,b)$-cores $\kappa$ in terms of their $a$-abacus diagrams. We use the Proposition~\ref{prop.core} one more time, but now looking at $\kappa$ as a $b$-core.

\begin{proposition}
\label{prop.sim}
Let $\kappa$ be an $a$-core with $d = \mathbf{abac}(\kappa)$ and consider positive integer number $b=sa+r$ with $0\le r<a$. Then $\kappa$ is a simultaneous $(a,b)$- core if and only if for any index $i$ between $1$ and $a-1$, one of the following is true:
\begin{enumerate}
\item $i \geq r$ and $d_{i} \leq d_{i-r} + s$,
\item $i<r$ and $d_{i} \leq d_{i+a-r}+s+1$.
\end{enumerate}
\end{proposition}

\begin{proof} ($\Rightarrow$)
Fix an index $i\in\{1,\ldots, a-1\}$ and consider all black beads on runner $i$ in nonnegative positions $k = 0,\ldots, d_i -1$. The corresponding SE-steps of $\partial(\kappa)$ enumerated by $j = i,\ i+a, \ldots ,\ i + a(d_i-1)$.

From Proposition~\ref{prop.core}, if $\kappa$ is a $b$- core, then for any positive SE-step $j$ in $\partial(\kappa)$ the step $j-b$ is also an SE-step.

If $i \ge r$, steps $j-b = (i-r) + a(k-s)$ with $k = 0,\ldots,d_i-1$ correspond to the black beads on the runner $(i-r)$ in positions $k = -s,\ -s+1,\ldots,\ d_i-s-1$, so the number of nonnegative black beads on the runner $(i-r)$ is greater or equal to $(d_i -s)$, and thus $d_{i} \leq d_{i-r} + s$.

Similarly, if $i<r$, steps $j-b = (i+a-r) + a(k-s-1)$ with $k = 0,\ldots,d_i-1$ correspond to the black beads on the runner $(i+a-r)$ in positions $k = -s-1,\ -s,\ldots,\ d_i-s-2$, so the number of nonnegative black beads on the runner $(i+a-r)$ is greater or equal to $(d_i - s-1)$, and thus $d_{i} \leq d_{i+a-r}+s+1$.

($\Leftarrow$) Conversely, consider any SE-step $j$ in $\partial(\lambda)$ with the corresponding black bead on runner $i$ and position $k < d_i$.

If $i \ge r$, the step $j-b$ corresponds to a bead on runner $(i-r)$ and position $(k-s)$. Since $k-s < d_i-s \le d_{i-r}$, that bead must be a black one and the step $(j-b)$ is an SE-step. 

If $i<r$, the step $j-b$ corresponds to a bead on runner $(i+a-r)$ and position $(k-s-1)$. Since $k-s-1 < d_i-s-1 \le d_{i+a-r}$, that bead must be a black one and the step $j-b$ is an SE-step.

\end{proof}

Together with Proposition~\ref{prop.distinct}, Proposition~\ref{prop.sim} gives a complete description of the simultaneous $(a,b)$-cores with distinct parts.

\section{Maximum size of $(a,a+1)$-cores with distinct parts.}
\label{sec.s=1}

Combining Proposition~\ref{prop.distinct} and Proposition~\ref{prop.sim} for the case $b=a+1$, we get the following result.

\begin{theorem}
\label{thm.s=1.abacus}
An $a$-core $\kappa$ is an $(a,a+1)$-core with distinct parts if and only if $d=\mathbf{abac}(\kappa)$ has entries $d_i \in \{0,1\}$ and the support set $\mathrm{supp}(d) = \{i: d_i = 1\}$ is an $a$-sparse set.
\end{theorem}
\begin{proof}
From Proposition~\ref{prop.distinct}, the fact that $\kappa$ has distinct parts is equivalent to the sparsity of the set $\mathrm{supp}(d)$.
Now, taking $s=1$ and $r=1$ in Proposition~\ref{prop.sim}, $\kappa$ is an $a+1$-core if and only if $d_i \le d_{i-1} + 1$ for all $i = 1,\ldots a-1$ (condition $d_0 \le d_{a-1} +2$ is always satisfied).

If $i$ is not in $\mathrm{supp}(d)$, then $d_i = 0$ and the equation $d_i \le d_{i-1} + 1$ is true.
If $i$ is in $\mathrm{supp}(d)$, then $d_{i-1} = 0$ because of the sparsity of $\mathrm{supp}(d)$ and so $d_i \le d_{i-1} + 1$ is equivalent to $d_i=1$.
\end{proof}

\begin{theorem}
\label{thm.s=1.fib}
The number of $(a,a+1)$-cores with distinct parts is equal to the Fibonacci number $F_{a+1}$.
\end{theorem}

\begin{proof}
By Theorem~\ref{thm.s=1.abacus}, all $(a,a+1)$-cores are in a bijection with $a$-sparse sets $S = \mathrm{supp}(\mathbf{abac}(\kappa))$. Let the number of $a$-sparse sets be $G_a$. Then, depending on whether an element $a-1$ is in a set, we can divide all $a$-sparse sets into two classes, so that $G_a = G_{a-1} + G_{a-2}$ and $G_1=1, \ G_2 =2$. Thus $G_a = F_{a+1}$.
\end{proof}

H.Xiong~\cite{Xiong.15} proved Theorem~\ref{thm.s=1.fib} together with conjectures about the largest size of $(a,a+1)$-cores with distinct parts and the number of such cores of maximal size (see Theorem~\ref{thm.s=1}). Here we provide another proof, which it is formulated in a different framework and which also will be useful for our future discussion.

\begin{theorem}
The largest size of an $(a,a+1)$-core with distinct parts is $\big\lfloor\frac{1}{3} \binom{a+1}{2}\big\rfloor$. Moreover, the core of maximal size is unique whenever $(a \mod 3)$ is $0$ or $2$ and there are two cores of maximal size when $(a \mod 3)$ is $1$.
\end{theorem}

\begin{proof}
Given an $(a,a+1)$-core $\kappa$ with distinct parts, take $d = \mathbf{abac}(\kappa)$, $\mathrm{supp}(d) = S = \{i_1,\ldots, i_n\}$, where $n = |\mathrm{supp}(d)|$ and indexes $0< i_1 <\ldots <i_n < a$. Denote the gaps between $i_{j}$ and $i_{j+1}$ as $g'_0 = i_1, \ g'_j = i_{j+1} - i_j -1$ for $j =1,\ldots, n-1$ and $g'_n = a-1-i_n$. 

Since $S$ is an $a$-sparse set, $g'_j \geq 1$ for $j=0,\ldots, n-1$, and thus we can instead consider nonnegative integer sequence $g_j = g'_j - 1$ for $j = 0,\ldots, n-1$ and $g_n = g'_n$. Notice that $\sum_{j=0}^n g'_j = a-n$ and $\sum_{j=0}^n g_j = a-2n$.

\begin{lemma}
Given $(a, a+1)$-core $\kappa$ with distinct parts,
\begin{equation}
\label{equation.size}
\mathrm{size}(\kappa) = \frac{1}{6} 3n (2a+1-3n) - \sum_{j=0}^n{j g_j}.
\end{equation}
\end{lemma}
\begin{proof}
Following the construction of an abacus diagram (see Fig.~\ref{figure.abacus}), each row $row_{n-j+1}$ of the partition $\kappa$ is bordered by an SE-step $se_j\in\partial(\kappa)$, which in turn corresponds to a black bead on the runner $i_j$ of the abacus diagram.
 
The length of that row $\kappa_{n-j+1}$ is determined by the number of NE-steps in $\partial(\kappa)$ before the step $se_j$. In the abacus diagram, those NE-steps would correspond to the white beads on runners $i = 0,\ldots,\ i_j-1$ in position 0, and the number of those white beads is equal to $\sum_{k=1}^{j-1} g'_k$.

Summing over all $j$,
\begin{multline*}
	\mathrm{size}(\kappa) = \sum_{j=1}^n \kappa_{n-j+1} =  \sum_{j=1}^n \sum_{k=0}^{j-1} g'_k = \sum_{j=0}^n (n-j) g'_j = n\sum_{j=0}^n g'_j - \sum_{j=0}^n j g'_j =\\ = n(a-n) - \frac{n(n-1)}{2} - \sum_{j=0}^n j g_j = \frac{1}{6} 3n (2a+1-3n) - \sum_{j=0}^n{j g_j}.
\end{multline*}
\end{proof}

Thus, to find a core of largest size, we maximize over $n$ and all nonnegative integer sequences $\{g_j\}_{j=0}^n$ with $\sum g_j = a-2n$.
\begin{equation*}
\max_{\kappa} \mathrm{size}(\kappa) = \max_n \max_{g_j \geq 0} \Big( \frac{1}{6} 3n (2a+1-3n) - \sum_{j=0}^n j g_j\Big) = \max_n \Big(\frac{1}{6} 3n (2a+1-3n) - \min_{g_j \geq 0} \sum_{j=0}^n j g_j\Big).
\end{equation*}

The minimum of $\sum j g_j$ over nonnegative sequences $\{g_j\}_{j=0}^n$ with $\sum g_j = a-2n$ is equal to 0 and is uniquely achieved when $g_0 = a-2n$ and $g_j = 0$ for $j \neq 0$. Thus,

\begin{equation*}
\max_{\kappa} \mathrm{size}(\kappa) = \max_n \Big(\frac{1}{6} 3n (2a+1-3n)\Big).
\end{equation*}

The parabola on the right-hand side has a rational supremum point at $\frac{2a+1}{6}$. Therefore, a value of $n$ that maximizes the function over integers is an integer point that has the minimal distance to $\frac{2a+1}{6}$.

\begin{enumerate}
\item When $(a\mod 3) = 0$, the maximum is achieved at the unique point $n = \frac{a}{3}$ and the value of the maximum is $\frac{1}{3} \binom{a+1}{2}$.
\item When $(a \mod 3) = 2$, the maximum is achieved at the unique point $n = \frac{a+1}{3}$ and the value of the maximum is $\frac{1}{3} \binom{a+1}{2}$.
\item When $(a \mod 3) = 1$, there are two integer points equally close to a number $\frac{2a+1}{6}$, which are $n_1 = \frac{a-1}{3}$ and $n_2 = \frac{a+2}{3}$. Both integers give the maximum value equal to $\frac{1}{3} \frac{(a-1)(a+2)}{2} = \big\lfloor \frac{1}{3} \binom{a+1}{2} \big\rfloor$.
\end{enumerate}
\end{proof}

In Section~\ref{sec.r=1} we give a generalization of the Proposition~\ref{thm.s=1.abacus} and provide another proof of part (4) of Theorem~\ref{thm.s=1}. Before we do that, however, we need to develop a notion of graded Fibonacci numbers in Section~\ref{sec.r=1}.


\section{Number of $(2k-1,2k+1)$-cores with distinct parts.}
\label{sec.r=2}

The number of $(2k-1,2k+1)$-cores with distinct parts was observed by T. Amdeberhan~\cite{Amdeberhan.15} and A. Straub~\cite{Straub.16} to be $2^{2k-2}$. That conjecture has been proved by  Yan, Qin, Jin and Zhou in \cite{YQJZ.16} using surprisingly deep arguments. Here, we present another perspective on $(2k-1,2k+1)$-cores using their connection with Dyck paths.

\begin{theorem}
The number of $(2k-1,2k+1)$-cores with distinct parts is equal to $2^{2k-2}$.
\end{theorem}

\begin{proof}
We will use the Dyck path interpretation of cores. For a Dyck path $\pi$, we denoted $\alpha(\pi)$ to be the set of ranks of the area boxes of $\pi$ (see Fig.~\ref{figure.anderson}). We will also make use of a standard notation $[n] = \{1,\ 2,\ldots,\ n\}$.

\begin{lemma}
Under the bijection $\mathbf{path}$, the set of $(2k-1,2k+1)$-cores with distinct parts maps to the set of $(2k-1,2k+1)$-Dyck paths $\pi$ such that $\alpha(\pi) \cap [2k-1]$ is a $(2k-1)$-sparse set. 
\end{lemma}

\begin{proof}[Proof of the lemma]
Given an $(2k-1,2k+1)$- core $\kappa$ with distinct parts and $\pi = \mathbf{path}(\kappa)$, the sparse set $\beta(\kappa)$ is equal to $\alpha(\pi)$. Therefore, we need to prove that the sparsity of $\alpha(\pi)$ is implied by the $(2k-1)$-sparsity of $\alpha(\pi)\cap [2k-1]$. 

Suppose for the sake of contradiction that $\alpha(\pi)\cap [2k-1]$ is $(2k-1)$-sparse and there are two elements $j$ and $j+1$ in $\alpha(\pi)$. Since $\alpha(\pi)$ is a $2k-1$-nested set, elements $(j~\text{mod}~2k-1)$, $(j+1~\text{mod}~2k-1)$ are in $\alpha(\pi)$, they are both in $[2k-1]$, and they differ by one (since there are no multiples of $2k-1$ in $\alpha(\pi)$). Therefore, we get a contradiction.
\end{proof}

Denote $T_k$ to be the upper triangle of $R_{2k-1,2k+1}$, i.e. $T_k$ consists of all boxes with positive rank in $R_{2k-1,2k+1}$ (see Fig.~\ref{figure.anderson}). Separate $T_k$ into three parts by a vertical line $x=k-1$ and a horizontal line $y=k+2$. Below the line $y=k+2$ the boxes of $T_k$ form a staircase-like shape \defn{$A$} that contains, among other boxes, the boxes of odd contents from $[2k-1]$. To the right of the line $x=k-1$ the boxes of $T_k$ form a staircase shape \defn{$B$} that contains the boxes of even contents from $[2k-1]$. Above $y=k+2$ and to the left of $x=k-1$ the boxes of $T_k$ form a square shape \defn{$C$} of size $k-1$.

Now we reflect the shape $A \cup C$ over the main diagonal $y=x$ and denote the resulting shape as \defn{$A^T \cup C^T$}. Put that shape to the right of $C \cup B$ to form a rectangular region \defn{$P_k$} $ = C \cup B \cup A^T \cup C^T$ (see Fig.~\ref{figure.2k-1_2k+1}).

\begin{figure}[t]
\includegraphics[scale=0.18]{conjecture_cores.jpg} \centering
\caption{The rectangle $P_{5}$ with a $C$-symmetric path $\zeta$. Here, $B(\zeta)=\{8\}$, $A^{T}(\zeta) = \{1,3,5,10,12,14,21\}$ and $C(\zeta) = C^{T}(\zeta) = \{19,30\}$. Also, $i(\zeta) = 1$ and $j(\zeta) = 0$.} \centering
\label{figure.2k-1_2k+1}
\end{figure}

We call a boundary between $B$ and $A^T$ to be the main diagonal of $P_k$. Consider the paths $\zeta$ from the SW corner of $P_k$ to the NE corner of $P_k$ consisting of N and E steps. Denote \defn{$C(\zeta)$} to be the set of contents of boxes below $\zeta$ in $C$, denote \defn{$B(\zeta)$} to be the set of contents of boxes below $\zeta$ in $B$, denote \defn{$A^T(\zeta)$} to be the set of contents of boxes \textit{above} $\zeta$ in $A^T$ and denote \defn{$C^T(\zeta)$} to be the set of contents of boxes \textit{above} $\zeta$ in $C^T$.

We call $\zeta$ to be $C$-symmetric when $C(\zeta) = C^T(\zeta)$ (see Fig.~\ref{figure.2k-1_2k+1} and compare with Fig.~\ref{figure.anderson}).

\begin{lemma}
The set of $C$-symmetric paths $\zeta$ in $P_k$ is in bijection \defn{$\phi$} with the set of  $(2k-1,2k+1)$-Dyck paths with $(2k-1)$-sparse $\alpha(\pi) \cap [2k-1]$. Moreover, $\alpha(\phi(\zeta)) = C(\zeta) \cup B(\zeta) \cup A^T(\zeta)$.
\end{lemma}

\begin{proof}[Proof of the lemma.]
We can define $\phi$ by the property above: $\phi(\zeta) = \pi$ if and only if $\alpha(\pi) = C(\zeta) \cup B(\zeta) \cup A^T(\zeta)$.
First, we need to make sure the map $\phi$ is well-defined, i.e. the set $\gamma(\zeta) := C(\zeta) \cup B(\zeta) \cup A^T(\zeta)$ is a $(2k-1,2k+1)$-nested set (i.e. check conditions (\ref{equation.nested})). 

Let $i \in \gamma(\zeta)$. If $i\in B(\zeta)$ or $i\in A^T(\zeta)$, conditions (\ref{equation.nested}) are satisfied since $B(\zeta)$ and $A^T(\zeta)$ are nested sets by construction. If $i\in C(\zeta) = C^T (\zeta)$, then $i-(2k+1) \in C(\zeta)\cup B(\zeta)$, since $C(\zeta)\cup B(\zeta)$ is $(2k+1)$-nested and $i-(2k-1) \in A^T(\zeta)\cup C^T(\zeta)$ since $A^T(\zeta)\cup C^T(\zeta)$ is $(2k-1)$-nested.

Second, we need to check that $\alpha(\phi(\zeta))\cap [2k-1] = \gamma(\zeta)\cap [2k-1]$ is $(2k-1)$-sparse. Consider $i \in (B(\zeta)\cup A(\zeta) )\cap [2k-1]$, and assume without loss of generality that $i\in B(\zeta)$. Then $i$ is even and it is bordering odd boxes $i-1$ and $i+1$ from E and S directions. Since $\zeta$ goes above the box $i$, it can not go below boxes $i-1$ and $i+1$ in $A^T$, and thus $i-1,i+1 \not\in A^T(\zeta)$.
\end{proof}

Now we want to count the number of $C$-symmetric paths $\zeta$ in $P_k$. If $C(\zeta)$ is non-empty, call $C$-shape of $\zeta$ to be the shape of the diagram under $\zeta$ in $C$, and $C^T$-shape of $\zeta$ is defined correspondingly. Denote \defn{$i(\zeta)$} to be the the width of the $C$-shape of $\zeta$ minus 1 (or the height of $C^T$-shape minus 1). Denote \defn{$j(\zeta)$} to be the height of the $C$-shape of $\zeta$ minus 1 (or the width of $C^T$-shape minus 1). 

The number of $C$-symmetric paths with fixed $i(\zeta) = i$ and fixed $j(\zeta) = j$ is the number of possible paths in $C$ times the number of possible paths in $B \cup A^T$, which is equal to $\binom{i+j}{i} \binom{k+1 + (k-3-i-j)}{k+1}$. If $C(\zeta)$ is empty, the number of paths is equal to $\binom{k+1 +(k-1)}{k+1}$. Thus, the total number of paths is
\begin{multline*}
\binom{2k}{k+1} + \sum_{i,j \geq 0} \binom{i+j}{i} \binom{2k-2-(i+j)}{k+1} = \binom{2k}{k+1} + \sum_{i\geq 0}\sum_{j' \geq 0} \binom{j'}{i} \binom{2k-2-j'}{k+1} = \\ 
= \binom{2k}{k+1} + \sum_{i\geq 0} \binom{2k-1}{k+2+i} = \binom{2k-1}{k} + \binom{2k-1}{k+1} + \sum_{i\geq 0} \binom{2k-1}{k+2+i} =\\ 
= \sum_{k \leq i' \leq 2k-1} \binom{2k-1}{i'} = \frac{1}{2} \sum_{0 \leq i' \leq 2k-1} \binom{2k-1}{i'} = 2^{2k-2}.
\end{multline*}
\end{proof}

\section{Graded Fibonacci numbers and $(a,as+1)$-cores with distinct parts.}
\label{sec.r=1}

Unfortunately, for general $b$ there is no easy way to combine Proposition~\ref{prop.distinct} and Proposition~\ref{prop.sim}. However it can be achieved for specific values of $b$.

\begin{theorem}
\label{thm.description}
Let $\kappa$ be an $a$-core and $b=as+1$ for some integer $s$. Then $\kappa$ is an $(a,b)$-core with distinct parts if and only if the abacus vector $d = \mathbf{abac}(\kappa)$ is $a$-sparse and $d_i \leq s$ for $i=1,\ldots, a-1$.
\end{theorem}

\begin{proof}
Similar to the proof of Theorem~\ref{thm.s=1.abacus}, we use Proposition~\ref{prop.distinct} to get the sparsity of $\mathrm{supp}(d)$. Moreover, using Proposition~\ref{prop.sim} with $r=1$ we see that $\kappa$ is an $(a,b)$-core if and only if $d_i \le d_{i-1} + s$ for all $i = 1,\ldots a-1$ (condition $d_0 \le d_{a-1} +s+1$ is automatically satisfied since $d_0 = 0$ for any abacus vector $d$).

If $i$ is not in $\mathrm{supp}(d)$, then $d_i = 0$ and the inequality $d_i \le d_{i-1} + s$ is true.
If $i$ is in $\mathrm{supp}(d)$, then $d_{i-1} = 0$ because of the sparsity of $\mathrm{supp}(d)$ and so $d_i \le d_{i-1} + s$ is equivalent to $1 \le d_i \le s$.
\end{proof}

We use similar argument for the case $b = as-1$.

\begin{theorem}
\label{thm.description2}
Let $\kappa$ be an $a$-core and $b=as-1$ for some integer $s$. Then $\kappa$ is an $(a,b)$-core with distinct parts if and only if the abacus vector $d = \mathbf{abac}(\kappa)$ is $a$-sparse, $d_i \leq s$ for $i\neq a-1$ and $d_{a-1} \le s-1$.
\end{theorem}

\begin{proof}
Again, we use Proposition~\ref{prop.distinct} to get the sparsity of $\mathrm{supp}(d)$, and use Proposition~\ref{prop.sim} with $r = a-1$ to get inequalities $d_i \le d_{i+1} + s$ for $i = 0,\ldots,\ a-2$ and $d_{a-1} \le d_0 + (s-1)$.

If $i$ is not in $\mathrm{supp}(d)$, then $d_i = 0$ and the inequality $d_i \le d_{i+1} + s$ is true.
If $i$ is in $\mathrm{supp}(d)$ and $i \neq a-1$, then $d_{i+1} = 0$ because of the sparsity of $\mathrm{supp}(d)$ and so $d_i \le d_{i-1} + s$ is equivalent to $1 \le d_i \le s$.
If $i = a-1$ and $i$ is in $\mathrm{supp}(d)$, note that $d_0$ is always 0, and thus $d_{a-1} \le d_0 + (s-1)$ equivalent to $1 \le d_{a-1} \le s-1$.
\end{proof}

For the further analysis we will need a generating function of the area statistic of cores $\kappa$.
We will call that function to be a graded Fibonacci number.

\begin{definition}
For two integers $a$ and $b$, the graded Fibonacci number is
\begin{equation}
\label{eq.fib.cores}
	F_{a,b}(q) = \sum_{\kappa} q^{area(\kappa)},
\end{equation}
where the sum is taken over all $(a,b)$-cores $\kappa$ with distinct parts.
\end{definition}

\begin{remark}
\label{remark.Catalan}
If the sum above was taken over all $(a,b)$-cores, we would have obtained a graded Catalan number (see~\cite{Loehr.03}).
\end{remark}

\begin{remark}
We do not require $a$ and $b$ to be coprime. Despite the fact that the sum would be infinite, the power series would converge for $|q| <1$. For the further analysis of Catalan numbers with $a$, $b$ not coprime, see~\cite{GMV.17}.
\end{remark}

\begin{remark}
\label{remark.s_to_infty}
When we set $s\to\infty$, the set of $(a,as+1)$-cores coveres the set of all $a$-cores with distinct parts. Thus we will also be interested in the limit of $F^{(s)}_a$ when $s\to\infty$.
\end{remark}

In the light of Theorem~\ref{thm.description} from here and until the end of the paper we will only consider the case $b=as+1$ (although all results that follow are applicable in the case $b = as-1$ with minor modifications). To shorten the notation, we define \defn{$F^{(s)}_a$} $ = F_{a, as+1}$.
It is helpful to rewrite the sum ~\eqref{eq.fib.cores} in terms of vectors $d=\mathbf{abac}(\kappa)$.
Denote the set of all $a$-sparse vectors $d = (d_0,\ldots, d_{a-1})$ with $d_i \le s$ as \defn{$\mathcal{A}^{(s)}_{a}$}.

\begin{theorem}
\begin{equation}
\label{equation.fib.d}
F^{(s)}_a(q) = \sum_{d\in\mathcal{A}^{(s)}_{a}} q^{\sum d_i}.
\end{equation}
\end{theorem}

\begin{proof}
From Proposition~\ref{thm.description}, the map $\mathbf{abac}\colon \kappa \rightarrow d$ is a bijection from the set of all $(a,b)$-cores $\kappa$ with distinct parts to the set of $a$-sparse vectors $d= (d_0, \ldots,\ d_{a-1})$ with $d_i \le s$, i.e. the set $\mathcal{A}^{(s)}_{a}$.

Also note that bijection $\mathbf{abac}$ sends the $area$ statistic of $\kappa$ to the sum $\sum_{i=0} ^{a-1} d_i$, since the number of rows in $\kappa$ is equal to the number of positive SE-steps of $\partial(\kappa)$, which in turn is equal to the number of nonnegative black beads in the abacus diagram of $\kappa$.

Thus,
\begin{equation*}
F^{(s)}_a(q) = \sum_{\kappa} q^{area(\kappa)} = \sum_{d = \mathbf{abac}(\kappa)} q^{\sum d_i} = \sum_{d\in\mathcal{A}^{(s)}_{a}} q^{\sum d_i}.
\end{equation*}
\end{proof}

Justification of the term ``graded Fibonacci numbers" comes from the proposition below. We will use a standard notation \defn{$\left(s\right)_q$} $ = 1+q+\cdots+q^{s-1} = \frac{1-q^s}{1-q}$. 

\begin{theorem}
\label{thm.recurrence}
Graded Fibonacci numbers $F^{(s)}_a (q)$ satisfy recurrence relation
\begin{equation}
\label{eq.fib.recurrence}
F^{(s)}_a (q) = F^{(s)}_{a-1} (q) + q \left(s\right)_q F^{(s)}_{a-2} (q)
\end{equation}
with initial conditions $F^{(s)}_{0} (q) = F^{(s)}_{1} (q) = 1$.
\end{theorem}

\begin{proof}
We divide the sum in (\ref{equation.fib.d}) into two parts: one over vectors $d$ with $a-1 \not\in \mathrm{supp}(d)$, and the other over vectors $d$ with $a-1 \in \mathrm{supp}(d)$.
\begin{multline*}
F^{(s)}_a(q) = \sum_{d \in \mathcal{A}^{(s)}_{a}} q^{\sum d_i} = \sum_{\substack{d\in \mathcal{A}^{(s)}_{a}\\ d_{a-1} = 0}} q^{\sum d_i} + \sum_{\substack{d\in \mathcal{A}^{(s)}_{a}\\ d_{a-1} \neq 0}} q^{\sum d_i} =
\\
= \sum_{d\in \mathcal{A}^{(s)}_{a-1}} q^{\sum d_i} + \sum_{d_{a-1} = 1}^{s} q^{d_{a-1}} \sum_{d \in \mathcal{A}^{(s)}_{a-2}} q^{\sum d_i} = F^{(s)}_{a-1} (q) + q \left(s\right)_q F^{(s)}_{a-2} (q).
\end{multline*}

For initial conditions, notice that $F^{(s)}_{1} (q) = 1$, since there is only one $(1,s+1)$-core, which is empty.
The number of $(2,2s+1)$-cores with distinct parts is equal to $s+1$, with corresponding $d_1 \in \{0, 1,\ldots, s\}$, and thus $F^{(s)}_{2} (q) = 1+q \left(s\right)_q$.

Following the recurrence we proved above, we can set $F^{(s)}_{0} (q) = 1$ for all $s$.
\end{proof}

We can now present another proof of Theorem~\ref{thm.E+} as a corollary of Theorem~\ref{thm.recurrence} by setting $q=1$. 

\begin{corollary}
The number of $(a, as+1)$-cores with distinct parts is equal to $F^{(s)}_a(1)$ and satisfies the recursive relation
\begin{equation*}
F^{(s)}_a (1) = F^{(s)}_{a-1} (1) + s F^{(s)}_{a-2} (1), \qquad F^{(s)}_{0}(1) = F^{(s)}_{1}(1) = 1.
\end{equation*}
In particular, $F^{(1)}_{a}(1) = F_{a+1}$ is a classical Fibonacci number.
\end{corollary}

\begin{remark}
In the limit $s \to\infty$, relation~\eqref{eq.fib.recurrence} has the form
\begin{equation*}
F^{(\infty)}_{a} (q) = F^{(\infty)}_{a-1} (q) + \frac{q}{1-q} F^{(\infty)}_{a-2} (q).
\end{equation*}
In light of Remark~\ref{remark.s_to_infty}, relation above is the recurrence for the generating function of $area(\kappa)$ over all $a$-cores with distinct parts.
\end{remark}

\begin{theorem}
\label{thm.decompose}
\begin{equation}
F^{(s)}_a (q) = \sum_{n=0}^{\lfloor a/2 \rfloor} \left(q \left(s\right)_{q}\right)^{n} \  \binom{a-n}{n}.
\end{equation}
\end{theorem}

\begin{proof}
For a fixed $a$-sparse support set $\mathbf{S}=\mathrm{supp}(d)$, the sum in (\ref{equation.fib.d}) is equal to 
\begin{equation}
\label{equation.fixed_S}
\sum_{\mathrm{supp}(d) = \mathbf{S}} q^{\sum d_i} = \prod_{i\in \mathbf{S}} \sum_{d_i=1}^s q^{d_i} = \left(q \left(s\right)_q\right)^{\left\vert{\mathbf{S}}\right\vert}.
\end{equation} 
For fixed $n = \left\vert{\mathbf{S}}\right\vert$, the number of possible $a$-sparse support sets $\mathbf{S}$ is the number  $n$-element subsets of $\{1, 2,\ldots, a-1\}$ such that no two elements are neighboring each other. The number of such subsets is equal to $\binom{a-n}{n}$.

Summing (\ref{equation.fixed_S}) over all $\mathbf{S}$,

\begin{equation*}
F^{(s)}_a(q) = \sum_{\mathbf{S}} \left(q \left(s\right)_q\right)^{\left\vert{\mathbf{S}}\right\vert} = \sum_{n=0}^{\lfloor a/2 \rfloor} \sum_{\left\vert{\mathbf{S}}\right\vert =n} \left(q \left(s\right)_q\right)^{n}=  \sum_{n=0}^{\lfloor a/2 \rfloor} \left(q \left(s\right)_{q}\right)^{n}  \  \binom{a-n}{n}.
\end{equation*}
\end{proof}

\begin{theorem}
The generating function for $F^{(s)}_a(q)$ with respect to $a$ is
\begin{equation}
G^{(s)} (x;q) := \sum_{a=0}^{\infty} x^a F^{(s)}_a (q) = \frac{1}{1-x-q\left(s\right)_q x^2}.
\end{equation}
\end{theorem}
\begin{proof}
We use the recurrence (\ref{eq.fib.recurrence}).
\begin{multline*}
G^{(s)} (x;q) = \sum_{a=0}^{\infty} x^a F^{(s)}_a (q) = 1 + x + \sum_{a=2}^{\infty} x^a F^{(s)}_a (q) =
\\
= 1+x+ \sum_{a=2}^{\infty} x^a F^{(s)}_{a-1} (q) + \sum_{a=2}^{\infty} x^a q \left(s\right)_q F^{(s)}_{a-2} (q) = 
\\
=1+x + x \sum_{a=1}^{\infty} x^a F^{(s)}_a (q) + q\left(s\right)_q x^2 \sum_{a=0}^{\infty} x^a F^{(s)}_a (q) = 
\\
=1+x +x \left(G^{(s)} (x;q) - 1\right) + q\left(s\right)_q x^2 G^{(s)} (x;q) =
\\
= 1 + \left(x+ q\left(s\right)_q x^2\right) G^{(s)} (x;q).
\end{multline*}
Thus
\begin{equation*}
G^{(s)} (x;q) = \frac{1}{1-x-q\left(s\right)_q x^2}.
\end{equation*}
\end{proof}

\begin{remark}
In the limit $s\to\infty$,
\begin{equation*}
F^{(\infty)}_{a} (q) = \sum_{n=0}^{\lfloor a/2 \rfloor} \left(\frac{q}{1-q}\right)^{n} \  \binom{a-n}{n}, \qquad G^{(\infty)} (x,q) = \frac{1}{1-x-\frac{q}{1-q} x^2}.
\end{equation*}
\end{remark}

Now we consider the case $s=1$ to give a proof of part (4) of Theorem~\ref{thm.s=1}.

\begin{remark}
When $s=1$,
\begin{equation}
\label{equation.s=1.sum}
F^{(1)}_{a} (q) = \sum_{n=0}^{\lfloor a/2 \rfloor} q^{n} \  \binom{a-n}{n}, \qquad G^{(1)} (x,q) = \frac{1}{1-x-qx^2},
\end{equation}
and
\begin{equation}
\label{equation.s=1.recurrence}
F^{(1)}_{a} (q) = F^{(1)}_{a-1} (q) + q F^{(1)}_{a-2} (q), \qquad F^{(1)}_{0}(q) = F^{(1)}_{1}(q) = 1.
\end{equation}
\end{remark}

\begin{theorem}
The total sum of the sizes and the average size of $(a,a+1)$-cores with distinct parts are, respectively, given by
\begin{equation}
\sum_{i+j+k=a+1} F_{i} F_{j} F_{k} \quad \mathrm{and} \quad \sum_{i+j+k=a+1} \frac{F_{i} F_{j} F_{k}}{F_{a+1}}.
\end{equation}
\end{theorem}

\begin{proof}
Denote $\Phi_a$ to be the total sum of sizes of $(a,a+1)$-cores with distinct parts. Since the generating function of Fibonacci numbers $\sum_{i=1}^\infty x^{i} F_{i}$ is equal to $\frac{x}{1-x-x^2}$, then in order to prove the theorem it is enough to show that the generating function $\Gamma(x) := \sum_{a=2}^\infty x^{a+1} \Phi_a$ is equal to 
\begin{equation}
\Gamma(x) \stackrel{?}{=} \sum_{a=2}^{\infty} x^{a+1} \sum_{i+j+k=a+1} F_{i} F_{j} F_{k} = \left( \sum_{i=1}^\infty x^i F_i \right)^3 = \left(\frac{x}{1-x-x^2}\right)^3.
\end{equation}
We use the equation (\ref{equation.size}) to find a formula for $\Phi_a$.

\begin{multline}
\label{equation.size.sum}
\Phi_{a} = \sum_{\kappa} \mathrm{size}(\kappa) = \sum_{n}\sum_{\substack{g \\ \sum g_{i} = a-2n}} \left[ \frac{1}{6} 3n (2a+1-3n) - \sum_{i=0}^n{i g_i} \right] =\\
= \sum_{n} \left[ \binom{a-n}{n} \frac{n (2a+1-3n)}{2} - \sum_{\substack{g \\ \sum g_{i} = a-2n}} \sum_{i=0}^{n} i g_{i} \right]
\end{multline}
To evaluate the double sum, we notice that taking $\lambda$ to be a partition with $\lambda = (1^{g_{1}} 2^{g_{2}}\ldots n^{g_{n}})$,

\begin{multline*}
\sum_{\sum g_{i} = a-2n} \sum_{i=0}^{n} i g_{i} = \sum_{\substack{\lambda_{1} \leq n \\ l(\lambda) \leq a-2n}} |\lambda| =\\
= \left(\text{number of}\ \lambda\ \text{in a rectangle}\ n\times (a-2n)\right) \cdot \left(\text{average size of}\ \lambda\ \text{in } n\times (a-2n)\right) .
\end{multline*}

Note that the number of partitions that fit rectangle $n\times (a-2n)$ is equal to the number of paths from the bottom-right corner of the rectangle to the top-left corner, and thus is equal to $\binom{a-n}{n}$. 

The average size of the partition is equal to half of the area of the rectangle $n\times (a-2n)$ because of the symmetry of partitions. Thus, the average size of $\lambda$ is equal to $\frac{n(a-2n)}{2}$.

Therefore,

\begin{equation*}
\sum_{\sum g_{i} = a-2n} \sum_{i=0}^{n} i g_{i}= \binom{a-n}{n}\  \frac{n(a-2n)}{2}.
\end{equation*}

Thus, the sum in (\ref{equation.size.sum}) evaluates to

\begin{equation*}
\Phi_a = \sum_{n} \binom{a-n}{n} \frac{n\left(a-(n-1)\right)}{2} = \frac{a}{2} \sum_{n} \binom{a-n}{n}\ n - \frac{1}{2} \sum_{n} \binom{a-n}{n}\ n(n-1).
\end{equation*}

Comparing it with (\ref{equation.s=1.sum}), the sum simplifies to
\begin{equation}
\label{eq.phi}
\Phi_a =\frac{a}{2} F'_{a} (1) - \frac{1}{2} F''_{a} (1),
\end{equation}
where the function $F_{a} (q)$ is a short-hand notation for $F^{(1)}_{a} (q)$, and the derivative $F'$ is taken with respect to $q$ (note that $F'_{0} (1) = F'_{1} (1) = 0$).
Using the expression for a generating function $G^{(1)} (x;q)$ in (\ref{equation.s=1.sum}),
\begin{multline*}
\Gamma (x) = \sum_{a=2}^{\infty} x^{a+1} \Phi_a \stackrel{(\ref{eq.phi})}{=} \frac{a}{2} \sum_{a=1}^\infty x^{a+1} F'_{a} (1) - \frac{1}{2} \sum_{a=0}^\infty x^{a+1} F''_{a} (1) =
\\
= \frac{x^2}{2} \sum_{a=1}^\infty a x^{a-1} F'_{a} (1) - \frac{x}{2} \sum_{a=0}^\infty x^{a} F''_{a} (1) =   \frac{x^2}{2} \frac{\partial^2 G^{(1)}(x;1)}{\partial x \partial q}-\frac{x}{2} \frac{\partial^2 G^{(1)}(x;1)}{\partial q^2} \stackrel{(\ref{equation.s=1.sum})}{=}
\\
\stackrel{(\ref{equation.s=1.sum})}{=} \frac{x^2}{2}\frac{2x(1-x-x^2) - 2x^2(-1-2x)}{(1-x-x^2)^3} - \frac{x}{2}\frac{2x^4}{(1-x-x^2)^3}
=\frac{x^3}{(1-x-x^2)^3}.
\end{multline*}
\end{proof}

\section{Bounce statistic and bigraded Fibonacci numbers.}
\label{section.bigraded}

In light of Remark \ref{remark.Catalan}, we can look at the summand of the bigraded Catalan numbers corresponding to the set of $(a,b)$-cores with distinct parts. There is a definition of bigraded Catalan numbers in terms of $(a,b)$-cores directly (see~\cite{ALW.16}), but the skew length statistic has rather complicated expression in terms of abacus vectors $d$. Nevertheless, we can use another pair of statistics on the set of $(a,b)$-Dyck paths in the case $b=as+1$, namely $area$ and $bounce$.

To define the $bounce$ statistic of a $(a,as+1)$-Dyck path $\pi$, first we present the construction of a bounce path for a Dyck path $\pi$ due to N.Loehr~\cite{Loehr.03}. 

We start at the point $(a,as+1)$ of the rectangle $R_{a,as+1}$ and travel in West direction until we hit an N-step (North step) of $\pi$. Denote $v_1$ to be the number of W-steps we did in the process, and travel $w_1 := v_1$ steps in the South direction. 

After that we travel in West direction until we hit an N-step of $\pi$ again. Denote $v_2$ to be the number of W-steps made this time, and travel in South direction $w_2 := v_2 +v_1$ steps if $s > 1$ or $w_2 := v_2$ steps if $s=1$. 

In general, on $k$-th iteration, after we travel $v_k \ge 0$ steps in West direction before hitting an N-step of $\pi$, we then travel in $S$ direction $w_k:=v_k+\ldots+v_{k-s+1}$ steps if $k\geq s$ or $w_k:=v_k+\ldots+v_1$ steps if $k < s$. The bounce path always stays above the main diagonal and eventually hits the point $(0,1)$, where the algorithm terminates (see~\cite{Loehr.03} for details). 

To calculate the \defn{bounce} statistic of $\pi$, each time our bounce path reaches an N-step $x$ of $\pi$ after traveling West, add up the number of squares to the left of $\pi$ and in the same row as $x$. We will call those rows \defn{bounce rows} (see Fig. \ref{figure.bounce}).

\begin{definition} \cite{Loehr.03}
Bigraded rational Catalan number is defined by the equation 
\begin{equation}
C_{a,as+1} (q,t) = \sum_{\pi} q^{area(\pi)} t^{bounce(\pi)},
\end{equation}
where the sum is taken over all $(a,as+1)$-Dyck paths $\pi$.
\end{definition}

\begin{figure}[t]
\includegraphics[scale=0.29]{bounce_path.png} \centering
\caption{$(4,13)$-Dyck path with $\alpha(\pi) = \{1,3,5,9\}$ and $e(\pi) = (3,0,1)$ together with its bounce path. The bounce rows are marked by arrows.} \centering
\label{figure.bounce}
\end{figure}

Following that definition, we restrict the sum to define bigraded Fibonacci numbers.

\begin{definition}
Bigraded rational Fibonacci number is defined by the equation
\begin{equation}
\label{definition.bigraded}
F^{(s)}_a (q,t) = \sum_{\pi} q^{area(\pi)} t^{bounce(\pi)},
\end{equation}
where the sum is taken over all $(a,as+1)$- Dyck paths such that $\kappa = \mathbf{core}(\pi)$ has distinct parts.
\end{definition}

\begin{remark} After the specialization $t=1$ bigraded Fibonacci numbers $F^{(s)}_a (q,1)$ are equal to graded $F^{(s)}_a (q) $ from the previous section. 
\end{remark}

Under the maps $\mathbf{core}$ and $\mathbf{abac}$, there is a correspondence between the Dyck paths $\pi$ in (\ref{definition.bigraded}) and abacus vectors $d \in \mathcal{A}^{(s)}_{a}$. Denote the bounce statistic on $\mathcal{A}^{(s)}_{a}$ to be a bounce statistic of the corresponding Dyck path $\pi$, i.e. $bounce(\mathbf{abac}(\mathbf{core}(\pi))) = bounce(\pi)$.

\begin{theorem}
Given an abacus vector $d\in\mathcal{A}^{(s)}_{a}$,
\begin{equation}
\label{bounce.abac}
bounce(d) = s\ \binom{a}{2} - \sum_{i=1}^{a-1} (a-i)d_i.
\end{equation}
\end{theorem}

\begin{proof}
Let $\pi$ be the corresponding Dyck path, i.e.~$\mathbf{abac}(\mathbf{core}(\pi)) = d$.
It is easier to consider the statistic $bounce'(\pi) = s \binom{a}{2} - bounce(\pi)$. Here $s \binom{a}{2}$ counts the total number of boxes in the upper triangle $T_{a,as+1}$, and thus $bounce'(\pi)$ counts the number of boxes in non-bounce rows to the left of $\pi$ plus the number of area boxes of $\pi$. Thus, we need to show
\begin{equation}
\label{equation.abacus.bounce}
bounce'(d) \stackrel{?}{=}  \sum_{i=1}^{a-1} (a-i)d_i.
\end{equation}

We will prove (\ref{equation.abacus.bounce}) by induction on $a$. Base case $a=1$ is straightforward. Assume now that (\ref{equation.abacus.bounce}) is true for any $a \leq k$ and consider the case $a=k+1$. 

According to Corollary~\ref{path.area.abac}, the area vector $e(\pi) = (d_1,\ldots,d_k)$. 

If $d_1 = e_1(\pi) = 0$, the first $s$ iterations of the bounce path algorithm yields values for W-steps $\nu_1=1,\ \nu_2 = \ldots = \nu_s = 0$ and the corresponding steps in South direction are $w_1 = w_2 = \ldots = w_s = 1$, ending at the point $(k,ks+1)$ (and after that point the values of $\nu_1,\ldots,\ \nu_s$ do not contribute to the bounce path). 

Thus the top $s$ rows of the upper triangle $T_{k+1,(k+1)s+1}$ are bounce rows and moreover there are no area boxes of $\pi$ in those rows. Therefore the top $s$ rows do not contribute anything to $bounce'(\pi)$, and we can safely erase them, reducing $a$ by one and reducing all indexes of $d$ by one. Denoting $d' = (d'_0,\ldots,d'_{k-1}) = (d_1,\ldots,d_k) = e(\pi)$,
\begin{equation*}
bounce'(d) = bounce'(d') = \sum_{i=1}^{k-1} (k-i)d'_i = \sum_{i=2}^{k} (k-(i-1))d_i = \sum_{i=1}^{k} ((k+1)-i)d_i.
\end{equation*}

If $d_1= e_1(\pi) >0$ (see Fig.\ref{figure.bounce}), then $d_2 =e_2(\pi) = 0$ because of the sparsity of $d$. The first $s$ iterations of the bounce path algorithm yields values for W-steps $\nu_1 =1, \ \nu_2= \ldots = \nu_{s-d_1} = 0,\ \nu_{s-d_1+1} = 1, \ \nu_{s-d_1+2}=\ldots = \nu_{2s-d_1} = 0$ with the corresponding steps in South direction equal to $w_1 =\ldots = w_{s-d_1} = 1$, $w_{s-d_1 +1} = \ldots = w_s = 2$, $w_{s+1} = \ldots = w_{2s-d_1} = 1$, ending at the point $(k-1,(k-1)s+1)$ (and after that point the values of $\nu_1,\ldots,\ \nu_{2s-d_1}$ do not contribute to the bounce path). 

For the top $2s$ rows of $T_{k+1,(k+1)s+1}$ the non-bounce rows appear exactly when the bounce path travels 2 steps in South direction, i.e.~when $w_i = 2$. Thus, there are $d_1$ non-bounce rows, each contributing $k-1$ boxes to statistic $bounce'$. Besides that, there are $d_1$ area boxes that also count towards $bounce'$. Thus, the contribution of the top $2s$ rows into $bounce'$ is equal to $d_1k$. Denoting $d' = (d'_0,\ldots,d'_{k-2}) = (d_2,\ldots,d_k)$,
\begin{equation*}
bounce'(d) = bounce'(d') + d_1k = \sum_{i=1}^{k-2} (k-1-i)d_{i+2} + d_1k = \sum_{i=1}^{k} ((k+1)-i)d_i.
\end{equation*}
\end{proof}

\begin{corollary}
\begin{equation}
\label{equation.fibonacci.abacus}
t^{-s\binom{a}{2}} F^{(s)}_a (q,t) =  \sum_{d\in \mathcal{A}^{(s)}_{a}} q^{\sum d_i} t^{-\sum (a-i) d_i} = \sum_{d\in \mathcal{A}^{(s)}_{a}} (qt^{-a})^{\sum d_i} t^{\sum i d_i} = \sum_{d\in \mathcal{A}^{(s)}_{a}} q^{\sum d_i} t^{-\sum i d_i}.
\end{equation}
\end{corollary}

\begin{proof}
First two equalities follow directly from (\ref{definition.bigraded}) and (\ref{bounce.abac}). For the last equality we use the symmetry of $\mathcal{A}^{(s)}_{a}$ under the reflection of indexes $0\mapsto 0, \ i \mapsto a-i$.
\end{proof}

From (\ref{equation.fibonacci.abacus}) it is easier to work with normalized polynomials
\begin{equation}
\label{equation.norm.fibonacci}
\tilde F^{(s)}_a (q,t) := t^{-s\binom{a}{2}} F^{(s)}_a (q,t^{-1}) = \sum_{d\in \mathcal{A}^{(s)}_{a}} q^{\sum d_i} t^{\sum i d_i} = \sum_{d\in \mathcal{A}^{(s)}_{a}} (qt^{a})^{\sum d_i} t^{-\sum i d_i}.
\end{equation}

\begin{remark}
In terms of Dyck paths, $\tilde F^{(s)}_a (q,t)$ is equal to the sum of $q^{area(\pi)} t^{bounce'(\pi)}$ over $(a,as+1)$-Dyck paths $\pi$ with $\mathbf{core}(\pi)$ having distinct parts.
\end{remark}

Using the simple expression of $\tilde F^{(s)}_a (q,t)$ in (\ref{equation.norm.fibonacci}), we prove recursive relations similar to Proposition~\ref{thm.recurrence}.

\begin{theorem}
Normalized bigraded Fibonacci numbers $\tilde F^{(s)}_a (q,t)$ satisfy the following relations:
\begeq
\label{recurrence1}
\tilde F^{(s)}_{a+1} (q,t) = \tilde F^{(s)}_{a} (q,t) + qt^a \left(s\right)_{qt^a} \tilde F^{(s)}_{a-1} (q,t)
\eneq
\begeq
\label{recurrence2}
\tilde F^{(s)}_{a+1} (q,t) = \tilde F^{(s)}_{a} (qt,t) + qt \left(s\right)_{qt} \tilde F^{(s)}_{a-1} (qt^2,t),
\eneq
with initial conditions $\tilde F^{(s)}_0 (q,t) =\tilde F^{(s)}_1(q,t) = 1$.
\end{theorem}
\begin{proof}
For equation \eqref{recurrence1} use the first sum in \eqref{equation.norm.fibonacci} and divide the set $\mathcal{A}^{(s)}_{a+1}$ into two parts corresponding to vectors $d$ with $d_{a} = 0$ and with $d_{a}>0$.
\begin{multline*}
\tilde F^{(s)}_{a+1} (q,t) = \sum_{d\in \mathcal{A}^{(s)}_{a+1}} q^{\sum d_i} t^{\sum i d_i} = 
\sum_{d' \in \mathcal{A}^{(s)}_{a}} q^{\sum d'_i} t^{\sum i d'_i} + \sum_{d_a=1}^s \left(qt^a\right)^{d_a} \sum_{d' \in \mathcal{A}^{(s)}_{a-1}} q^{\sum d'_i}  t^{\sum i d'_i} = \\
= \tilde F^{(s)}_{a} (q,t) + qt^a \left(s\right)_{qt^a} \tilde F^{(s)}_{a-1} (q,t).
\end{multline*}
For equation \eqref{recurrence2} use the second sum in \eqref{equation.norm.fibonacci} and again divide $\mathcal{A}^{(s)}_{a+1}$ into two parts corresponding to vectors $d$ with $d_{a} = 0$ and with $d_{a}>0$.
\begin{multline*}
\tilde F^{(s)}_{a+1} (q,t) = \sum_{d\in \mathcal{A}^{(s)}_{a+1}} (qt^{a+1})^{\sum d_i} t^{-\sum i d_i} =\\
= \sum_{d' \in \mathcal{A}^{(s)}_{a}} \left((qt)t^{a-1}\right)^{\sum d'_i} t^{-\sum i d'_i} + \sum_{d_a=1}^s (qt^{a+1} t^{-a})^{d_a} \sum_{d' \in \mathcal{A}^{(s)}_{a-1}} \left((qt^2)t^{a-2}\right)^{\sum d'_i} t^{-\sum i d'_i} =\\
= \tilde F^{(s)}_{a} (qt,t) + qt \left(s\right)_{qt} \tilde F^{(s)}_{a-1} (qt^2,t)
\end{multline*}
\end{proof}

\begin{remark}
Setting $s=1$ and $a\to\infty$, the recurrence \eqref{recurrence2} gives
\begeq
\tilde F^{(1)}_{\infty} (q,t) = \tilde F^{(1)}_{\infty} (qt,t) + qt \tilde F^{(1)}_{\infty} (qt^2,t),
\eneq
which is an Andrews $q$-difference equation related to Rogers-Ramanujan identities (see~\cite{Andrews.86}).
\end{remark}
