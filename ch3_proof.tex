\section{Proof of Theorem~\ref{theorem.main2}}
\label{section.proof main2}
%%%%%%%%%%%%%%%%%%%%%%%%%%%%%%%%%%%%%%%%%%%%%%%%%%%%%%%%%

In this section, we provide the proof of Theorem~\ref{theorem.main2}.

%%%%%%%%%%%%%%%%%%%%%%%%%%%%%%%%%%%%%%%%%%%%%%%%%%%%%%%%%
\subsection{Preliminaries}

We use the fact from \cite{Haiman.1989} that taking only elements smaller or equal to $i+1$ from the word $\mathbf{b}$ and 
applying the mixed insertion corresponds to taking only the part of the tableau $\mathbf{T}$ with elements $\leqslant i+1$.
Thus, it is enough to prove the theorem for a ``truncated'' word $\mathbf{b}$ without any letters greater than $i+1$.
To shorten the notation, we set $j= i+1$ in this appendix. We sometimes also restrict to just the letters $i$ and $j$
in a word $w$. We call this the \defn{$\{i,j\}$-subword} of $w$.

First, in Lemma~\ref{lemma.main} we justify the notion of the reading word $\mathrm{rw}(\textbf{T})$ and provide the 
reason to use a bracketing rule on it. After that, in Section~\ref{section.main.proof} we prove that the action of the 
crystal operator $f_i$ on $\mathbf{b}$ corresponds to the action of $f_i$ on $\mathbf{T}$ after the insertion.

Given a word $\mathbf{b}$, we apply the crystal bracketing rule for its $\{i,j\}$-subword and globally declare the 
rightmost unbracketed $i$ in $\mathbf{b}$ (i.e. the letter the crystal operator $f_i$ acts on) to be a bold $i$.
Insert the letters of $\mathbf{b}$ via Haiman insertion to obtain the insertion tableau $\mathbf{T}$. During this process, 
we keep track of the position of the bold $i$ in the tableau via the following rules. When the bold $i$ from $\mathbf{b}$ is 
inserted into $\mathbf{T}$, it is inserted as the rightmost $i$ in the first row of $\mathbf{T}$ since by definition it is 
unbracketed in $\mathbf{b}$ and hence cannot bump a letter $j$. From this point on, the tableau $\mathbf{T}$ has a 
\defn{special} letter $i$ and we track its position:

\begin{enumerate}
\item If the special $i$ is unprimed, it is always the rightmost $i$ in its row. When a letter $i$ is bumped from this row, 
only one of the non-special letters $i$ can be bumped, unless the special $i$ is the only $i$ in the row. When the non-diagonal 
special $i$ is bumped from its row to the next row, it will be inserted as the rightmost $i$ in the next row.
\item When the diagonal special $i$ is bumped from its row to the column to its right, it is inserted as the bottommost $i'$ 
in the next column.
\item If the special $i$ is primed, it is always the bottommost $i'$ in its column. When a letter $i'$ is bumped from this
column, only one of the non-special letters $i'$ can be bumped, unless the special $i'$ is the only $i'$ in the column. 
When the primed special $i$ is bumped from its column to the next column, it is inserted as the bottommost $i'$ in the 
next column.
\item When $i$ is inserted into a row with the special unprimed $i$, the rightmost $i$ becomes special.
\item When $i'$ is inserted into a column with the special primed $i$, the bottommost primed $i$ becomes special.
\end{enumerate}

\begin{lemma}
\label{lemma.main}
Using the rules above, after the insertion process of $\mathbf{b}$, the special $i$ in $\mathbf{T}$ is the same as 
the rightmost unbracketed $i$ in the reading word $\mathrm{rw}(\mathbf{T})$ (i.e. the definition of the bold $i$ in
$\mathbf{T}$). Moreover, the number of unbracketed letters $i$ in $\mathbf{b}$ is equal to the number of unbracketed 
letters $i$ in $\mathrm{rw}(\mathbf{T})$.
\end{lemma}

\begin{proof}
First, note that since both the number of letters $i$ and the number of letters $j$ are equal in $\mathbf{b}$ and 
$\mathrm{rw}(\mathbf{T})$, the fact that the number of unbracketed letters $i$ is the same implies that the number of
unbracketed letters $j$ must also be the same. We use induction on $1 \leqslant s \leqslant h$, where the letters 
$b_1 \ldots b_s$ of $\mathbf{b}=b_1 b_2 \ldots b_h$ have been inserted using Haiman mixed insertion with the above 
rules. That is, we check that at each step of the insertion algorithm the statement of our lemma stays true. 

The induction step is as follows: Consider the word $b_1 \ldots b_{s-1}$ with a corresponding 
insertion tableau $\mathbf{T}^{(s-1)}$. 
If the bold $i$ in $\mathbf{b}$ is not in $b_1\ldots b_{s-1}$, then $\mathbf{T}^{(s-1)}$ does not contain a special letter $i$.
Otherwise, by induction hypothesis assume that the bold $i$ in $b_1\ldots b_{s-1}$  by the above rules corresponds to the 
special $i$ in $\mathbf{T}^{(s-1)}$, that is, it is in the position corresponding to the rightmost unbracketed $i$ in the 
reading word $\mathrm{rw}(\mathbf{T}^{(s-1)})$. Then we need to prove that for $b_1 \ldots b_s$, the special $i$ 
in $\mathbf{T}^{(s-1)}$ ends up in the position corresponding to the rightmost unbracketed $i$ in the reading word of
$\mathbf{T}^{(s)} = \mathbf{T}^{(s-1)} \leftsquigarrow b_s$.
We also need to verify that the second part of the lemma remains true for $\mathbf{T}^{(s)}$.

Remember that we are only considering ``truncated'' words $\mathbf{b}$ with all letters $\leqslant j$.

\smallskip

\noindent
\textbf{Case 1.}
Suppose $b_s = j$. 
In this case $j$ is inserted at the end of the first row of $\mathbf{T}^{(s-1)}$, and $\mathrm{rw}(\mathbf{T}^{(s)})$ has 
$j$ attached at the end. Thus, both statements of the lemma are unaffected.

\smallskip

\noindent
\textbf{Case 2.} 
Suppose $b_s = i$ and $b_s$ is unbracketed in $b_1 \ldots b_{s-1} b_s$.
Then there is no special $i$ in tableau $\mathbf{T}^{(s-1)}$, and $b_s$ might be the bold $i$ of the word $\mathbf{b}$.
Also, there are no unbracketed letters $j$ in $b_1 \ldots b_{s-1}$, and thus all $j$ in $\mathrm{rw}(\mathbf{T}^{(s-1)})$ 
are bracketed. Thus, there are no letters $j$ in the first row of $\mathbf{T}^{(s-1)}$, and $i$ is inserted in the first 
row of $\mathbf{T}^{(s-1)}$, possibly bumping the letter $j'$ from column $c$ into an empty column $c+1$ in the process.
Note that if $j'$ is bumped, moving it to column $c+1$ of $\mathbf{T}^{(s)}$ does not change the reading word, 
since column $c$ of $\mathbf{T}^{(s-1)}$ does not contain any primed letters other than $j'$.
The reading word of $\mathbf{T}^{(s)}$ is thus the same as $\mathrm{rw}(\mathbf{T}^{(s-1)})$ except for an additional 
unbracketed $i$ at the end. The number of unbracketed letters $i$ in both $\mathrm{rw}(\mathbf{T}^{(s)})$ and 
$b_1 \ldots b_{s-1} b_s$ is thus increased by one compared to $\mathrm{rw}(\mathbf{T}^{(s-1)})$ and $b_1 \ldots b_{s-1}$.
If $b_s$ is the bold $i$ of the word $\mathbf{b}$, the special $i$ of tableau $\mathbf{T}^{(s)}$ is the rightmost $i$ on the 
first row and corresponds to the rightmost unbracketed $i$ in $\mathrm{rw}(\mathbf{T}^{(s)})$. 

\smallskip

\noindent
\textbf{Case 3.} 
Suppose $b_s = i$ and $b_s$ is bracketed with a $j$ in the word $b_1\ldots b_{s-1}$. 
In this case, according to the induction hypothesis, $\mathrm{rw}(\mathbf{T}^{(s-1)})$ has an unbracketed $j$. There are 
two options. 

\smallskip

\noindent
\textbf{Case 3.1.} 
If the first row of $\mathbf{T}^{(s-1)}$ does not contain $j$, $b_s$ is inserted at the end of the first row of $\mathbf{T}^{(s-1)}$, 
possibly bumping $j'$ in the process. 
Regardless, $\mathrm{rw}(\mathbf{T}^{(s)})$ does not change except for attaching an $i$ at the end (see Case 2). 
This $i$ is bracketed with one unbracketed $j$ in $\mathrm{rw}(\mathbf{T}^{(s)})$. The special $i$ (if there was one 
in $\mathbf{T}^{(s-1)}$) does not change its position and the statement of the lemma remains true.

\smallskip

\noindent
\textbf{Case 3.2.} 
If the first row of $\mathbf{T}^{(s-1)}$ does contain a $j$, inserting $b_s$ into $\mathbf{T}^{(s-1)}$ bumps $j$ 
(possibly bumping $j'$ beforehand) into the second row, where $j$ is inserted at the end of the row. 
So, if the first row contains $n \geqslant 0$ elements $i$ and $m \geqslant 1$ elements $j$, the reading 
word $\mathrm{rw}(\mathbf{T}^{(s-1)})$ ends with $\ldots i^n j^m$, and $\mathrm{rw}(\mathbf{T}^{(s)})$ ends with 
$\ldots j i^{n+1} j^{m-1}$.  Thus, the number of unbracketed letters $i$ does not change and if there was a special $i$ 
in the first row, it remains there and it still corresponds to the rightmost unbracketed $i$ in $\mathrm{rw}(\mathbf{T}^{(s)})$.

\smallskip

\noindent
\textbf{Case 4.} 
Suppose $b_s < i$. 
Inserting $b_s$ could change both the primed reading word and unprimed reading word of $\mathbf{T}^{(s-1)}$. 
As long as neither $i$ nor $j$ is bumped from the diagonal, we can treat primed and unprimed changes separately. 

\smallskip

\noindent
\textbf{Case 4.1.} 
Suppose neither $i$ nor $j$ is not bumped from the diagonal during the insertion.
This means that there are no transitions of letters $i$ or $j$ between the primed and the unprimed parts of the reading word.
Thus, it is enough to track the bracketing relations in the unprimed reading word; the bracketing relations in the primed 
reading word can be verified the same way via the transposition. After we make sure that the number of unbracketed letters 
$i$ and $j$ changes neither in the primed nor unprimed reading word, it is enough to consider the case when the 
special $i$ is unprimed, since the case when it is primed can again be checked using the transposition.
To avoid going back and forth, we combine these two processes together in each subcase to follow.

\smallskip

\noindent
\textbf{Case 4.1.1.} 
If there are no letters $i$ and $j$ in the bumping sequence, the unprimed $\{i,j\}$-subword of $\mathrm{rw}(\mathbf{T}^{(s)})$ 
is the same as in $\mathrm{rw}(\mathbf{T}^{(s-1)})$. The special $i$ (if there is one) remains in its position, and thus the 
statement of the lemma remains true.

\smallskip

\noindent
\textbf{Case 4.1.2.} 
Now consider the case when there is a $j$ in the bumping sequence, but no $i$. 
Let that $j$ be bumped from the row $r$. Since there is no $i$ bumped, row $r$ does not contain any letters $i$. 
Thus, bumping $j$ from row $r$ to the end of row $r+1$ does not change the $\{i,j\}$-subword
of $\mathrm{rw}(\mathbf{T}^{(s-1)})$, so the statement of the lemma remains true.

\smallskip

\noindent
\textbf{Case 4.1.3.} 
Consider the case when there is an $i$ in the bumping sequence. Let that $i$ be bumped from the row $r$. 

\smallskip

\noindent
\textbf{Case 4.1.3.1.} 
If there is a (non-diagonal) $j$ in row $r+1$, it is bumped into row $r+2$  ($j'$ may have been bumped in the process). 
Note that in this case the $i$ bumped from row $r$ could not have been a special one. 
If there are $n \geqslant 0$ elements $i$ and $m \geqslant 1$ elements $j$ in row $r$, the part of the reading 
word $\mathrm{rw}(\mathbf{T}^{(s-1)})$ with $\ldots i^n j^m i \ldots$ changes to $\ldots j i^{n+1} j^{m-1} \ldots$ in $\mathrm{rw}(\mathbf{T}^{(s)})$. 
The bracketing relations remain the same, and if row $r+1$ contained a special $i$, it would remain there and would 
correspond to the rightmost $i$ in $\mathrm{rw}(\mathbf{T}^{(s)})$.

\smallskip

\noindent
\textbf{Case 4.1.3.2.} 
If there are no letters $j$ in row $r+1$, and $j'$ in row $r+1$ does not bump a $j$, the 
$\{i,j\}$-subword does not change and the statement of the lemma remains true.

\smallskip

\noindent
\textbf{Case 4.1.3.3.} 
Now suppose there are no letters $j$ in row $r+1$ and $j'$ from row $r+1$ bumps a $j$ from another row.
This can only happen if, before the $i$ was bumped, there was only one $i$ in row $r$ of $\mathbf{T}^{(s-1)}$, there is 
a $j'$ right below it, and there is a $j$ in the column to the right of $i$ and in row $r' \leqslant r$. 

If $r'=r$, then after the insertion process, $i$ and $j$ are bumped from row 
$r$ to row $r+1$. Since there was only one $i$ in row $r$ and there are no letters $j$ in row $r+1$, the $\{i,j\}$-subword 
of $\mathrm{rw}(\mathbf{T}^{(s-1)})$ does not change and the statement of the lemma remains true.

Otherwise $r' < r$. Then there are no letters $i$ in row $r'$ and by assumption there is no letter $j$ in row $r+1$. 
Thus, moving $i$ to row $r+1$ and moving $j$ to the row $r'+1$ does not change the $\{i,j\}$-subword of 
$\mathrm{rw}(\mathbf{T}^{(s-1)})$ and the statement of the lemma remains true.

\smallskip

\noindent
\textbf{Case 4.2.} 
Suppose $i$ or $j$ (or possibly both) are bumped from the diagonal in the insertion process.

\smallskip

\noindent
\textbf{Case 4.2.1.} 
Consider the case when the insertion sequence ends with $\quad\cdots \rightarrow z \rightarrow j [j']$ with $z<i$ 
and possibly $ \rightarrow j$ right after it. Let the bumped diagonal $j$ be in column $c$. 
Then columns $1,2, \ldots, c$ of $\mathbf{T}^{(s-1)}$ could only contain elements $\leqslant z$, except for the $j$ on the 
diagonal. Thus, the bumping process just moves $j$ from the unprimed reading word to the primed reading word 
without changing the overall order of the $\{i,j\}$-subword.

\smallskip

\noindent
\textbf{Case 4.2.2.} 
Consider the case when the insertion sequence ends with $\quad \cdots \rightarrow i' \rightarrow i \rightarrow j[j']$ and 
possibly $\rightarrow j$. Let the bumped diagonal $j$ be in row (and column) $r$. Note that $r$ must be the last row of 
$\mathbf{T}^{(s-1)}$. Then $i$ has to be bumped from row $r-1$ (and, say, column $c$) and $i'$ also has to be in row $r-1$ 
(moreover, it has to be the only $i'$ in column $c-1$). 
Also, since there are no letters $j'$ in column $c$ (otherwise it would be in row $r$, which is impossible), 
bumping $i'$ to column $c$ does not change the $\{i,j\}$-subword of $\mathrm{rw}(\mathbf{T}^{(s-1)})$. 
Note that after $i'$ moves to column $c$, there are no $i'$ or $j'$ in columns $1,\ldots, r$, and thus priming $j$ and 
moving it to column $r+1$ does not change the $\{i,j\}$-subword.
If the last row $r$ contains $n$ elements $j$, the $\{i,j\}$-subword of $\mathbf{T}^{(s-1)}$ contains $\ldots j^n i \ldots$ 
and after the insertion it becomes $\ldots j i j^{n-1} \ldots$, where  the left $j$ is from the primed subword. 
Thus, the number of bracketed letters $i$ does not change. 
Also, if we moved the special $i$ in the process, it could only have been the bumped $i'$. Its position in the reading 
word is unaffected.

\smallskip

\noindent
\textbf{Case 4.2.3.} 
The case when the insertion sequence does not contain $i'$, does not bump $i$ from the diagonal, but contains $i$ and 
bumps $j$ from the diagonal is analogous to the previous case.

\smallskip

\noindent
\textbf{Case 4.2.4.} 
Suppose both $i$ and $j$ are bumped from the diagonal. 
That could only be the case with diagonal $i$ bumped from row (and column) $r$, 
bumping another letter $i$ from the row $r$ and column $r+1$, 
and bumping $j$ from row (and column) $r+1$ (and possibly bumping $j$ to row $r+2$ at the end). 
Let the number of letters $i'$ in column $r+1$ be $n$ and let the number of letters $j$ in row $r+1$ be $m$. 

\smallskip

\noindent
\textbf{Case 4.2.4.1} 
Let $m\geqslant 2$. Then the $\{i,j\}$-subword of $\mathrm{rw}(\mathbf{T}^{(s-1)})$ contains $\ldots i^n j^m ii \ldots$ 
and after the insertion it becomes $\ldots j i^{n+1} j i j^{m-2} \ldots$. 
The number of unbracketed letters $i$ stays the same. Since $m \geqslant  2$, the special $i$ of $\mathbf{T}^{(s-1)}$ 
could not  have been involved in the bumping procedure. However, the special $i$ might have been the bottommost $i'$ in 
column $r+1$ of $\mathbf{T}^{(s-1)}$, and after the insertion the special $i$ would still be the bottommost $i'$ in column 
$r+1$ and would correspond to the rightmost unbracketed $i$ in $\mathrm{rw}(\mathbf{T}^{(s)})$:
\begin{equation*}
\young(\cdot\cdot\iprime\cdot,:ii\cdot,::jj) \quad \mapsto \quad
\young(\cdot\cdot\iprime\cdot,:\cdot\iprime\cdot,::i\jprime,:::j)
\end{equation*}

\smallskip

\noindent
\textbf{Case 4.2.4.2.} 
Let $m=1$. Then the $\{i,j\}$-subword of $\mathbf{T}^{(s-1)}$ contains $\ldots i^n j ii \ldots$ and after the insertion it 
becomes $\ldots j i^{n+1} i$. The number of unbracketed letters $i$ stays the same.
If the special $i$ was in row $r$ and column $r+1$, then after the insertion it becomes a diagonal one, and it would still 
correspond to the rightmost unbracketed $i$ in $\mathrm{rw}(\mathbf{T}^{(s)})$.

\smallskip

\noindent
\textbf{Case 4.2.5.} 
Suppose only $i$ is bumped from the diagonal (let that $i$ be on row and column $r$). 
Note that there cannot be an $i'$ in column $r$.

\smallskip

\noindent
\textbf{Case 4.2.5.1.} 
Suppose $i$ from the diagonal bumps another $i$ from column $r+1$ and row $r$. 
In that case there are no letters $j$ in row $r+1$. No letters $j$ or $j'$ are affected and thus the 
$\{i,j\}$-subword of $\mathbf{T}^{(s)}$ does not change, and the special $i$ in $\mathbf{T}^{(s)}$ (if there is one) 
still corresponds to the rightmost unbracketed $i$ in $\mathrm{rw}(\mathbf{T}^{(s)})$.

\smallskip

\noindent
\textbf{Case 4.2.5.2.} 
Suppose $i$ from the diagonal bumps $j'$ from column $r+1$ and row $r$. 
Note that $j'$ must be the only $j'$ in column $r+1$. 
Suppose also that there is one $j$ in row $r+1$. 
Denote the number of letters $i'$ in column $r+1$ of $\mathbf{T}^{(s-1)}$ by $n$. 
If there is a $j$ in row $r+1$ of $\mathbf{T}^{(s-1)}$, then the $\{i,j\}$-subword of $\mathbf{T}^{(s-1)}$ contains 
$\ldots i^n jji \ldots$ and after the insertion it becomes $\ldots ji^{n+1}j \ldots$.
If there is no $j$ in row $r+1$ of $\mathbf{T}^{(s-1)}$, then the $\{i,j\}$-subword of $\mathbf{T}^{(s-1)}$ contains 
$\ldots i^n ji \ldots$ and after the insertion it becomes $\ldots ji^{n+1} \ldots$.
The number of unbracketed letters $i$ is unaffected. If the special $i$ of $\mathbf{T}^{(s-1)}$ was the bottommost $i'$ in 
column $r+1$ of $\mathbf{T}^{(s-1)}$, after the insertion the special $i$ is still the bottommost $i'$ in column 
$r+1$ and corresponds to the rightmost unbracketed $i$ in $\mathrm{rw}(\mathbf{T}^{(s)})$.
\end{proof}

\begin{corollary}
\label{corollary.f annihilate}
	\begin{equation*}
		f_i (\mathbf{b}) = \mathbf{0} \quad \text{if and only if} \quad f_i (\mathbf{T}) = \mathbf{0}.
	\end{equation*}
\end{corollary}

%%%%%%%%%%%%%%%%%%%%%%%%%%%%%%%%%%%%%%%%%%%%%%%%%%%%%%%%%
\subsection{Proof of Theorem~\ref{theorem.main2}}
\label{section.main.proof}
By Lemma~\ref{lemma.main}, the cell $x$ in the definition of the operator $f_i$ corresponds to the bold $i$ in the 
tableau $\mathbf{T}$. Furthermore, we know how the bold $i$ moves during the insertion procedure. 
We assume that the bold $i$ exists in both $\mathbf{b}$ and $\mathbf{T}$, meaning that $f_i(\mathbf{b}) \neq \mathbf{0}$
and $f_i(\mathbf{T}) \neq \mathbf{0}$ by Corollary~\ref{corollary.f annihilate}.
We prove Theorem~\ref{theorem.main2} by induction on the length of the word $\mathbf{b}$.

\smallskip

\noindent
\textbf{Base.} 
Our base is for words $\mathbf{b}$ with the last letter being a bold $i$ (i.e. rightmost unbracketed $i$). 
Let $\mathbf{b} = b_1 \ldots b_{h-1} b_h$ and $f_i(\mathbf{b}) = b_1 \ldots b_{h-1} b'_h$, where $b_h = i$ and $b'_h = j$. 
Denote the mixed insertion tableau of $b_1 \ldots b_{h-1}$ as $\mathbf{T}_0$, the insertion tableau of 
$b_1 \ldots b_{h-1} b_h$ as $\mathbf{T}$, and the insertion tableau of $b_1 \ldots b_{h-1} b'_h$ as $\mathbf{T}'$. 
Note that $\mathbf{T}_0$ does not  have letters $j$ in the first row. If the first row of $\mathbf{T}_0$ ends with $\ldots j'$, 
then the first row of $\mathbf{T}$ ends with $\ldots \mathbf{i} j'$ and the first row of $\mathbf{T}'$ ends with $\ldots j' j$. 
If the first row of $\mathbf{T}_0$ does not contain $j'$, the first row of $\mathbf{T}$ ends with $\ldots \mathbf{i}$ and the first 
row of $\mathbf{T}'$ ends with $\ldots j$, and the cell $x_S$ is empty. 
In both cases $f_i(\mathbf{T}) = \mathbf{T}'$. 

\smallskip

\noindent
\textbf{Induction step.} 
Now, let $\mathbf{b} = b_1 \ldots b_h$ with operator $f_i$ acting on the letter $b_s$ in $\mathbf{b}$ with $s < h$. 
Denote the mixed insertion tableau of $b_1 \ldots b_{h-1}$ as $\mathbf{T}$ and the insertion tableau of 
$f_i(b_1 \ldots b_{h-1})$ as $\mathbf{T}'$. By induction hypothesis, we know that $f_i(\mathbf{T}) = \mathbf{T}'$. 
We want to show that $f_i(\mathbf{T} \leftsquigarrow b_h) = \mathbf{T}' \leftsquigarrow b_h$.
In Cases 1-3 below, we assume that the bold letter $i$ is unprimed. Since almost all results from the case with unprimed 
$i$ are transferrable to the case with primed bold $i$ via the transposition of the tableau $\mathbf{T}$, we just need to cover 
the differences in Case 4.

\smallskip

\noindent
\textbf{Case 1.} 
Suppose $\mathbf{T}$ falls under Case (1) of the rules for $f_i$: the bold $i$ is in the non-diagonal cell $x$ in row $r$ and 
column $c$ and the cell $x_E$ in the same row and column $c+1$ has content $j'$. Consider the insertion path of $b_h$.

\smallskip

\noindent
\textbf{Case 1.1.} 
If the insertion path of $b_h$ in $\mathbf{T}$ contains neither cell $x$ nor cell $x_E$, the insertion path of $b_h$ in 
$\mathbf{T}'$ also does not contain cells $x$ and $x_E$. Thus, $f_i(\mathbf{T} \leftsquigarrow b_h) 
= \mathbf{T}' \leftsquigarrow b_h$.

\smallskip

\noindent
\textbf{Case 1.2.} 
Suppose that during the insertion of $b_h$ into $\mathbf{T}$, the bold $i$ is row-bumped by an unprimed element 
$d < i$ or is column-bumped by a primed element $d' \leqslant i'$. 
This could only happen if the bold $i$ is the unique $i$ in row $r$ of $\mathbf{T}$. 
During the insertion process, the bold $i$ is inserted into row $r+1$. 
Since there are no letters $i$ in row $r$ of $\mathbf{T}'$, inserting $b_h$ into $\mathbf{T}'$ inserts $d$ in cell $x$, 
bumps $j'$ to cell $x_E$, and bumps $j$ into row $r+1$. Thus we are in a situation similar to the induction base. 
It is easy to check that row $r+1$ does not contain any letters $j$ in $\mathbf{T}$.
If it contains $j'$, this $j'$ is bumped back into 
row $r+1$. Similar to the induction base, $f_i(\mathbf{T} \leftsquigarrow b_h) = \mathbf{T}' \leftsquigarrow b_h$.

\smallskip

\noindent
\textbf{Case 1.3.} 
Suppose that during the insertion of $b_h$ into $\mathbf{T}$, an unprimed $i$ is inserted into row $r$. 
Note that in this case, row $r$ in $\mathbf{T}$ must contain a $j$ (or else the $i$ from row $r$ would not be the 
rightmost unbracketed $i$ in $\mathrm{rw}(\mathbf{T})$). Thus inserting $i$ into row $r$ in $\mathbf{T}$ shifts the bold $i$ to 
column $c+1$, shifts $j'$ to column $c+2$ and bumps $j$ to row $r+1$. Inserting  $i$ into row $r$ in $\mathbf{T}'$ shifts 
$j'$ to column $c+1$ with a $j$ to the right of it, and bumps $j$ into row $r+1$. 
Thus $f_i(\mathbf{T} \leftsquigarrow b_h) = \mathbf{T}' \leftsquigarrow b_h$.

\smallskip

\noindent
\textbf{Case 1.4.}
Suppose that during the insertion of $b_h$ into $\mathbf{T}$, the $j'$ in cell $x_E$ is column-bumped by a primed 
element $d'$ and the cell $x$ is unaffected. Note that in order for $\mathbf{T} \leftsquigarrow b_h$ to be a valid 
primed tableau, $i$ must be smaller than $d'$, and thus $d'$ could only be $j'$. On the other hand, $j'$ cannot be inserted 
into column $c+1$ of $\mathbf{T}'$ in order for  $\mathbf{T}' \leftsquigarrow b_h$ to be a valid primed tableau. 
Thus this case is impossible.

\smallskip

\noindent
\textbf{Case 2.} 
Suppose tableau $\mathbf{T}$ falls under Case (2a) of the crystal operator rules for $f_i$. 
This means that for a bold $i$ in cell $x$ (in row $r$ and column $c$) of tableau $\mathbf{T}$, the cell $x_E$ has 
content $j$ or is empty and cell $x_S$ is empty. Tableau $\mathbf{T}'$ has all the same elements as $\mathbf{T}$, 
except for a $j$ in the cell $x$. We are interested in the case when inserting $b_h$ into either $\mathbf{T}$ or 
$\mathbf{T}'$ bumps the element from cell $x$.

\smallskip

\noindent
\textbf{Case 2.1.} 
Suppose that the non-diagonal bold $i$ in $\mathbf{T}$ (in row $r$) is row-bumped by an unprimed element 
$d < i$ or column-bumped by a primed element $d' < j'$. Element $d$ (or $d'$) bumps the bold $i$ into row $r+1$ of 
$\mathbf{T}$, while in $\mathbf{T}'$ (since there are no letters $i$ in row $r$ of $\mathbf{T}'$) it bumps $j$ from cell $x$ 
into row $r+1$. Thus we are in the situation of the induction base and $f_i(\mathbf{T} \leftsquigarrow b_h) 
= \mathbf{T}' \leftsquigarrow b_h$.

\smallskip

\noindent
\textbf{Case 2.2.}
Suppose $x$ is a non-diagonal cell in row $r$, and during the insertion of $b_h$ into $\mathbf{T}$, an unprimed $i$ 
is inserted into the row $r$. In this case, row $r$ in $\mathbf{T}$ must contain a letter $j$.
The insertion process shifts the bold $i$ one cell to the right in $\mathbf{T}$ and bumps a $j$ into row $r+1$, while 
in $\mathbf{T}'$ it just bumps $j$ into the row $r+1$. We end up in Case (2a) of the crystal operator rules for $f_i$
with bold $i$ in the cell $x_E$.

\smallskip

\noindent
\textbf{Case 2.3.}
Suppose that during the insertion of $b_h$ into $\mathbf{T}'$, the $j$ in the non-diagonal cell $x$ is column-bumped 
by a $j'$. This means that $j'$ was previously bumped from column $c-1$ and row $\geqslant r$. Thus the cell 
$x_{SW}$ (cell to the left of an empty $x_{S}$) is non-empty. Moreover, right before inserting $j'$ into the column $c$, 
the cell $x_{SW}$ has content $< j'$. Inserting $j'$ into column $c$ of $\mathbf{T}$ just places $j'$ into the empty cell $x_S$.
Inserting $j'$ into column $c$ of $\mathbf{T}'$ places $j'$ into $x$, and bumps $j$ into the empty cell $x_S$.
Thus, we end up in Case (2c) of the crystal operator rules after the insertion of $b_h$ with $y = x_S$.

\smallskip

\noindent
\textbf{Case 2.4.} 
Suppose that $x$ in $\mathbf{T}$ is a diagonal cell (in row $r$ and column $r$) and that it is row-bumped by an 
element $d<i$. Note that in this case there cannot be any letter $j$ in row $r+1$.
Also, since $d$ is inserted into cell $x$, there cannot be any letters $i'$ in columns $1,\ldots, r$, and thus there 
cannot be any letters $j'$ in column $r+1$ (otherwise the $i$ in cell $x$ would not be bold).
The bumped bold $i$ in tableau $\mathbf{T}$ is inserted as a primed bold $i'$ into the cell $z$ of column $r+1$.

\smallskip

\noindent
\textbf{Case 2.4.1.}
Suppose that there are no letters $i$ in column $r+1$ of $\mathbf{T}$.
In this case, the cell $z$ in $\mathbf{T}$ either contains $j$ (and then that $j$ would be bumped to the next row) or is empty.
Inserting $b_h$ into tableau $\mathbf{T}'$ bumps the diagonal $j$ in cell $x$, which is inserted as a $j'$ into 
cell $z$, possibly bumping $j$ after that.
Thus, $\mathbf{T} \leftsquigarrow b_h$ falls under Case (2a) of the ``primed'' crystal rules with the bold $i'$ in cell 
$z$ (note that there cannot be any $j'$ in cell $(z*)_E$ of the tableau $(\mathbf{T} \leftsquigarrow b_h)*$). 
Since $\mathbf{T} \leftsquigarrow b_h$ and $\mathbf{T}' \leftsquigarrow b_h$ differ only by the cell $z$, 
$f_i(\mathbf{T} \leftsquigarrow b_h) = \mathbf{T}' \leftsquigarrow b_h$.

\smallskip

\noindent
\textbf{Case 2.4.2.}
Suppose that there is a letter $i$ in cell $z$ of column $r+1$ of $\mathbf{T}$.
Note that cell $z$ can only be in rows $1, \ldots, r-1$ and thus $z_{SW}$ contains an element $< i$.
Thus, during the insertion process of $b_h$ into $\mathbf{T}$, diagonal bold $i$ from cell $x$ is inserted as bold $i'$ 
into cell $z$, bumping the $i$ from cell $z$ into cell $z_S$ (possibly bumping $j$ afterwards).
On the other hand, inserting $b_h$ into $\mathbf{T}'$ bumps the diagonal $j$ from cell $x$ into cell $z_S$ 
as a $j'$ (possibly bumping $j$ afterwards).
Thus, $\mathbf{T} \leftsquigarrow b_h$ falls under Case (1) of the ``primed'' crystal rules with the bold $i'$ in cell 
$z$, and so $f_i(\mathbf{T} \leftsquigarrow b_h) = \mathbf{T}' \leftsquigarrow b_h$.

\smallskip

\noindent
\textbf{Case 2.5.} 
Suppose that $x$ is a diagonal cell (in row $r$ and column $r$) and that during the insertion of $b_h$ into $\mathbf{T}$, an 
unprimed $i$ is inserted into row $r$. In this case, the content of cell $x_E$ has to be $j$ and the diagonal 
cell $x_{ES}$ must be empty. Inserting $i$ into row $r$ of $\mathbf{T}$ bumps a $j$ from cell $x_E$ into 
cell $x_{ES}$. On the other hand, inserting $i$ into row $r$ of $\mathbf{T}'$ bumps a $j$ from the diagonal cell $x$, 
which in turn is inserted as a $j'$ into cell $x_E$, which bumps $j$ from cell $x_E$ into cell $x_{ES}$.
Thus, $\mathbf{T} \leftsquigarrow b_h$ falls under Case (2b) of the crystal rules with bold $i$ in cell $x_E$ and 
$y= x_{ES}$, and so $f_i(\mathbf{T} \leftsquigarrow b_h) = \mathbf{T}' \leftsquigarrow b_h$.

\smallskip

\noindent
\textbf{Case 3.}
Suppose that $\mathbf{T}$ falls under Case (2b) or (2c) of the crystal operator rules.
That means $x_E$ has content $j$ or is empty and $x_S$ 
has content $j'$ or $j$. There is a chain of letters $j'$ and $j$ in $\mathbf{T}$ starting from $x_S$ and ending on a box $y$.
According to the induction hypothesis, $y$ is either on the diagonal and has content $j$ or $y$ is not on the diagonal and 
has content $j'$. The tableau $\mathbf{T}' = f_i (\mathbf{T})$ has $j'$ in cell $x$ and $j$ in cell $y$.
We are interested in the case when inserting $b_h$ into $\mathbf{T}$ affects cell $x$ or affects some element of the chain.
Let $r_x$ and $c_x$ be the row and the column index of cell $x$, and $r_y$, $c_y$ are defined accordingly.
Note that during the insertion process, $j'$ cannot be inserted into columns $c_y,\ldots, c_x$ and $j$ cannot be
inserted into rows $r_x +1,\ldots, r_y$, since otherwise $\mathbf{T} \leftsquigarrow b_h$ would not be a primed tableau.

\smallskip

\noindent
\textbf{Case 3.1.} 
Suppose the bold $i$ in cell $x$ (of row $r_x$ and column $c_x$) of $\mathbf{T}$ is row-bumped by 
an unprimed element $d < i$ or column-bumped by a primed element $d' < i$.
Note that in this case, bold $i$ in row $r_x$ is the only $i$ in this row, so row $r_x+1$ cannot contain 
any letter $j$. Therefore the content of $x_S$ must be $j'$.
In tableau $\mathbf{T}$, the bumped bold $i$ is inserted into cell $x_S$ and $j'$ is bumped from cell $x_S$ 
into column $c_x+1$, reducing the chain of letters $j'$ and $j$ by one. Notice that since $x_E$ either contains a $j$ 
or is empty,  $j'$ cannot be bumped into a position to the right of $x_S$, so Case (1) of the crystal 
rules for $\mathbf{T} \leftsquigarrow b_h$ cannot occur.  As for $\mathbf{T}'$, inserting $d$ into row $r_x$ (or inserting $d'$ into 
column $c_x$) just bumps $j'$ into column $c_x+1$, thus reducing the length of the chain by one in that tableau as well.
Note that in the case when the length of the chain is one (i.e. $y=x_S$), we would end up in Case (2a) of the crystal 
rules after the insertion. Otherwise, we are still in Case (2b) or (2c). 
In both cases, $f_i(\mathbf{T} \leftsquigarrow b_h) = \mathbf{T}' \leftsquigarrow b_h$.

\smallskip

\noindent
\textbf{Case 3.2.}
Suppose a letter $i$ is inserted into the same row as $x$ (in row $r_x$).
In this case, $x_E$ must contain a $j$ (otherwise the bold $i$ would not be in cell $x$).
After inserting $b_h$ into $\mathbf{T}$, the bold $i$ moves to cell $x_E$ (note that there cannot be a $j'$ to the right of 
$x_E$) and $j$ from $x_E$ is bumped to cell $x_{ES}$, thus the chain now starts at $x_{ES}$.
As for  $\mathbf{T}'$, inserting $i$ into the row $r_x$ moves $j'$ from cell $x$ to the cell $x_E$ and moves $j$ 
from cell $x_E$ to cell $x_{ES}$. Thus, $f_i(\mathbf{T} \leftsquigarrow b_h) = \mathbf{T}' \leftsquigarrow b_h$.

\smallskip

\noindent
\textbf{Case 3.3.}
Consider the chain of letters $j$ and $j'$ in $\mathbf{T}$.
Suppose an element of the chain $z \neq x,y$ is row-bumped by an element $d < j$ or is column-bumped 
by an element $d'<j'$. The bumped element $z$ (of row $r_z$ and column $c_z$) must be a ``corner'' element of the chain, 
i.e. in $\mathbf{T}$ the contents of boxes must be $c(z)=j', \ c(z_E) = j$ and $c(z_S)$ must be either $j$ or $j'$.
Therefore, inserting $b_h$ into $\mathbf{T}$ bumps $j'$ from box $z$ to box $z_E$ and bumps $j$ from
box $z_E$ to box $z_{ES}$, and inserting $b_h$ into $\mathbf{T}'$ has exactly the same effect.
Thus, there is still a chain of letters $j$ and $j'$ from $x_S$ to $y$ in $\mathbf{T}$ and $\mathbf{T}'$, and 
$f_i(\mathbf{T} \leftsquigarrow b_h) = \mathbf{T}' \leftsquigarrow b_h$.

\smallskip

\noindent
\textbf{Case 3.4.}
Suppose $\mathbf{T}$ falls under Case (2c) of the crystal rules (i.e. $y$ is not a diagonal cell) and during the insertion of 
$b_h$ into $\mathbf{T}$, $j'$ in cell $y$ is row-bumped (resp. column-bumped) by an element $d<j'$ (resp. $d'<j'$).
Since $y$ is the end of the chain of letters $j$ and $j'$, $y_S$ must be empty.
Also, since it is bumped, the content of $y_E$ must be $j$.
Thus, inserting $b_h$ into $\mathbf{T}$ bumps $j'$ from cell $y$ to cell $y_E$ and bumps $j$ from cell 
$y_E$ into row $r_y+1$ and column $\leqslant c_y$. On the other hand, inserting $b_h$ into $\mathbf{T}'$ bumps $j$ from
cell $y$ into row $r_y+1$ and column $\leqslant c_y$. The chain of letters $j$ and $j'$ now ends at $y_E$ and 
$f_i(\mathbf{T} \leftsquigarrow b_h) = \mathbf{T}' \leftsquigarrow b_h$.

\smallskip

\noindent
\textbf{Case 3.5.}
Suppose $\mathbf{T}$ falls under Case (2b) of the crystal rules (i.e. $y$ with content $j$ is a diagonal cell) and during 
the insertion of $b_h$ into $\mathbf{T}$, $j$ in cell $y$ is row-bumped by an element $d < j$.
In this case, the cell $y_E$ must have content $j$. Thus, inserting $b_h$ into $\mathbf{T}$ bumps $j$ from cell 
$y$ (making it $j'$) to cell $y_E$ and bumps $j$ from cell $y_E$ to the diagonal cell $y_{ES}$. On the other hand,
inserting $b_h$ into $\mathbf{T}'$ has exactly the same effect. The chain of letters $j$ and $j'$ now ends at 
the diagonal cell $y_{ES}$, so $\mathbf{T}\leftsquigarrow b_h$ falls under Case (2b) of the crystal rules and 
$f_i(\mathbf{T} \leftsquigarrow b_h) = \mathbf{T}' \leftsquigarrow b_h$.

\smallskip

\noindent
\textbf{Case 4.} 
Suppose the bold $i$ in tableau $\mathbf{T}$ is a primed $i$.
We use the transposition operation on $\mathbf{T}$, and the resulting tableau $\mathbf{T}^*$ falls under one of the cases 
of the crystal operator rules. When $b_h$ is inserted into $\mathbf{T}$, we can easily translate the insertion process to 
the transposed tableau $\mathbf{T}^*$ so that $[\mathbf{T}^* \leftsquigarrow (b_h+1)'] = [\mathbf{T} \leftsquigarrow b_h]^*$: 
the letter $(b_h+1)'$ is inserted into the first column of $\mathbf{T}^*$, and all other insertion rules stay exactly same, 
with one exception -- when the diagonal element $d'$ is column-bumped from the diagonal cell of 
$\mathbf{T}^*$, the element $d'$ becomes $(d-1)$ and is inserted into the row below.
Notice that the primed reading word of $\mathbf{T}$ becomes an unprimed reading word of $\mathbf{T}^*$.
Thus, the bold $i$ in tableau $\mathbf{T}^*$ corresponds to the rightmost unbracketed $i$ in the \textit{unprimed} 
reading word of $\mathbf{T}^*$. Therefore, everything we have deduced in Cases 1-3 from the fact that bold $i$ is in the 
cell $x$ will remain valid here. Given $f_i(\mathbf{T}^*) = \mathbf{T}'^*$, we want to make sure that 
$f_i(\mathbf{T}^* \leftsquigarrow (b_h+1)') = \mathbf{T}'^* \leftsquigarrow (b_h+1)'$.

The insertion process of $(b_h+1)'$ into $\mathbf{T}^*$ falls under one of the cases above and the proof of 
$f_i(\mathbf{T}^* \leftsquigarrow (b_h+1)') = \mathbf{T}'^* \leftsquigarrow (b_h+1)'$ is exactly the same as the proof in 
those cases. We only need to check the cases in which the diagonal element might be affected differently in the 
insertion process of $(b_h+1)'$ into $\mathbf{T}^*$ compared to the insertion process of $(b_h+1)'$ into $\mathbf{T}'^*$.
Fortunately, this never happens: in Case 1 neither $x$ nor $x_E$ could be diagonal elements; in Cases 2 and 3 $x$ 
cannot be on the diagonal, and if $x_E$ is on diagonal, it must be empty.
Following the proof of those cases, $f_i(\mathbf{T}^* \leftsquigarrow (b_h+1)') = \mathbf{T}'^* \leftsquigarrow (b_h+1)'$.

%%%%%%%%%%%%%%%%%%%%%%%%%%%%%%%%%%%%%%%%%%%%%%%%%%%%%%%%%
\section{Proof of Theorem~\ref{theorem.main3}}
\label{section.proof main3}
%%%%%%%%%%%%%%%%%%%%%%%%%%%%%%%%%%%%%%%%%%%%%%%%%%%%%%%%%

This appendix provides the proof of Theorem~\ref{theorem.main3}. In this section we set $j=i+1$.
We begin with two preliminary lemmas.

%%%%%%%%%%%%%%%%%%%%%%%%%%%%%%%%%%%%%%%%%%%%%%%%%%%%%%%%%
\subsection{Preliminaries}

\begin{lemma}
\label{lemma.chains}
	Consider a shifted tableau $\mathbf{T}$.
	\begin{enumerate}
	\item Suppose tableau $\mathbf{T}$ falls under Case (2c) of the $f_i$ crystal operator rules, that is, there is a chain of 
	letters $j$ and $j'$ starting from the bold $i$ in cell $x$ and ending at $j'$ in cell $x_H$. 
	Then for any cell $z$ of the chain containing $j$, the cell $z_{NW}$ contains $i$.
	\item  Suppose tableau $\mathbf{T}$ falls under Case (2b) of the $f_i$ crystal operator rules, that is, there is a chain of 
	letters $j$ and $j'$ starting from the bold $i$ in cell $x$ and ending at $j$ in the diagonal cell $x_H$. 
	Then for any cell $z$ of the chain containing $j$ or $j'$, the cell $z_{NW}$ contains $i$ or $i'$ respectively. 
	\end{enumerate}
\end{lemma}
\Yboxdim 13pt
\begin{equation*}
\young(\cdot\cdot\cdot\cdot\cdot\cdot\cdot\boldi,:\cdot\cdot\cdot iii\jprime,::\cdot\cdot\jprime jjj,:::\cdot\jprime) \qquad
\young(\cdot\cdot\cdot\cdot\iprime\boldi,:\cdot\iprime ii\jprime,::i\jprime jj,:::j)
\end{equation*}

\begin{proof}
The proof of the first part is based on the observation that every $j$ in the chain must be bracketed with some $i$ in the 
reading word $\mathrm{rw}(\mathbf{T})$. 
Moreover, if the bold $i$ is located in row $r_x$ and rows $r_x, r_x+1,\ldots, r_z$ contain $n$ letters $j$,
then rows $r_x, r_x +1,\ldots, r_z-1$ must contain exactly $n$ non-bold letters $i$.
To prove that these elements $i$ must be located in the cells to the North-West of the cells containing $j$, we
proceed by induction on $n$. When we consider the next cell $z$ containing $j$ in the chain that must be bracketed, 
notice that the columns $c_z, c_z+1,\ldots, c_x$ already contain an $i$, and thus we must put the next $i$ in column 
$c_z -1$; there is no other row to put it than $r_z-1$. Thus, $z_{NW}$ must contain an $i$.

This line of logic also works for the second part of the lemma. We can show that for any cell $z$ of the chain containing 
$j$, the cell $z_{NW}$ must contain an $i$. As for cells $z$ containing $j'$, we can again use the fact that the corresponding 
letters $j$ in the primed reading word of $\mathbf{T}$ must be bracketed.
Notice that these letters $j'$ cannot be bracketed with unprimed letters $i$, since all unprimed letters $i$ are already bracketed 
with unprimed letters $j$. Thus, $j'$ must be bracketed with some $i'$ from a column to its left.
Let columns $1,2, \ldots, c_z$ contain $m$ elements $j'$.
Using the same induction argument as in the previous case, we can show that $z_{NW}$ must contain $i'$. 
\end{proof}

Next we need to figure out how $y$ in the raising crystal operator $e_i$ is related to the lowering operator rules
for $f_i$.
 
\begin{lemma}
\label{lemma.y}
Consider a pair of tableaux $\mathbf{T}$ and $\mathbf{T}' = f_i(\mathbf{T})$.
	\begin{enumerate}
	\item If tableau $\mathbf{T}$ (in case when bold $i$ in $\mathbf{T}$ is unprimed) or $\mathbf{T}^*$ 
	(if bold $i$ is primed) falls under Case (1) of the $f_i$ crystal operator rules, then cell $y$ of 
	the $e_i$ crystal operator rules is cell $x_E$ of $\mathbf{T}'$ or $(\mathbf{T}')^*$, respectively.
	
	\item If tableau $\mathbf{T}$ (in case when bold $i$ in $\mathbf{T}$ is unprimed) or $\mathbf{T}^*$ 
	(if bold $i$ is primed) falls under Case (2a) of the $f_i$ crystal operator rules, then cell $y$ of 
	the $e_i$ crystal operator rules is located in cell $x$ of $\mathbf{T}'$ or $(\mathbf{T}')^*$, respectively.
	
	\item If tableau $\mathbf{T}$ falls under Case (2b) of the $f_i$ crystal operator rules, then cell $y$ of 
	the $e_i$ crystal operator rules is cell $x^*$ of $(\mathbf{T}')^*$.
	
	\item If tableau $\mathbf{T}$ (in case when bold $i$ in $\mathbf{T}$ is unprimed) or $\mathbf{T}^*$ 
	(if bold $i$ is primed) falls under Case (2c) of the $f_i$ crystal operator rules, then cell $y$ of 
	the $e_i$ crystal operator rules is cell $x_H$ of $\mathbf{T}'$ or $(\mathbf{T}')^*$, respectively.
	\end{enumerate}
\end{lemma}

\begin{proof}
In all the cases above, we need to compare reading words $\mathrm{rw}(\mathbf{T})$ and $\mathrm{rw}(\mathbf{T}')$. 
Since $f_i$ affects at most two boxes of $\mathbf{T}$, it is easy to track how the reading word $\mathrm{rw}(\mathbf{T})$ 
changes after applying $f_i$. We want to check where the bold $j$ under $e_i$ ends up in $\mathrm{rw}(\mathbf{T}')$ 
and in $\mathbf{T}'$, which allows us to determine the cell $y$ of the $e_i$ crystal operator rules.

\smallskip

\noindent
\textbf{Case 1.1.} 
Suppose $\mathbf{T}$ falls under Case (1) of the $f_i$ crystal operator rules, that is, the bold $i$ in cell $x$ is to the left of 
$j'$ in cell $x_E$. Furthermore, $f_i$ acts on $\mathbf{T}$ by changing the content of $x$ to $j'$ and by changing the content 
of $x_E$ to $j$. In the reading word $\mathrm{rw}(\mathbf{T})$, this corresponds to moving the $j$ corresponding to $x_E$ 
to the left and changing the bold $i$ (the rightmost unbracketed $i$) corresponding to cell $x$ to $j$ (that then corresponds
to $x_E$). Moving a bracketed $j$ in $\mathrm{rw}(\mathbf{T})$ to the left does not change the $\{i,j\}$ bracketing, and 
thus the $j$ corresponding to $x_E$ in $\mathrm{rw}(\mathbf{T}')$ is still the leftmost unbracketed $j$. Therefore, this $j$ 
is the bold $j$ of $\mathbf{T}'$ and is located in cell $x_E$.

\smallskip

\noindent
\textbf{Case 1.2.}
Suppose the bold $i$ in $\mathbf{T}$ is primed and $\mathbf{T}^*$ falls under Case (1) of the $f_i$ crystal operator 
rules. After applying lowering crystal operator rules to $\mathbf{T}^*$ and conjugating back, the bold primed $i$ in cell 
$x^*$ of $\mathbf{T}$ changes to an unprimed $i$, and the unprimed $i$ in cell $(x^*)_S$ of $\mathbf{T}$ changes 
to $j'$. In terms of the reading word of $\mathbf{T}$, it means moving the bracketed $i$ (in the unprimed reading word) 
corresponding to $(x^*)_S$ to the left so that it corresponds to $x^*$, and then changing the bold $i$ (in the 
primed reading word) corresponding to $x^*$ into the letter $j$ corresponding to $(x^*)_S$.
The first operation does not change the bracketing relations between $i$ and $j$, and thus the leftmost unbracketed $j$ 
in $\mathrm{rw}(\mathbf{T}')$ corresponds to $(x^*)_S$. Hence the bold unprimed $j$ is in cell $x_E$ of 
$(\mathbf{T}')^*$.

\smallskip

\noindent
\textbf{Case 2.1.}
If $\mathbf{T}$ falls under Case (2a) of the $f_i$ crystal operator rules, $f_i$ just changes the content of $x$ from $i$ to $j$.
The rightmost unbracketed $i$ in the reading word of $\mathbf{T}$ changes to the leftmost unbracketed $j$ in 
$\mathrm{rw}(\mathbf{T}')$. Thus, the bold $j$ in $\mathrm{rw}(\mathbf{T}')$ corresponds to cell $x$.

\smallskip

\noindent
\textbf{Case 2.2.}
The case when $\mathbf{T}^*$ falls under Case (2a) of the $f_i$ crystal operator rules is the same as the previous case.

\smallskip

\noindent
\textbf{Case 3.}
Suppose $\mathbf{T}$ falls under Case (2b) of $f_i$ crystal operator rules. Then there is a chain starting from cell 
$x$ (of row $r_x$ and column $c_x$) and ending at the diagonal cell $z$ (of row and column $r_z$) consisting of 
elements $j$ and $j'$. Applying $f_i$ to $\mathbf{T}$ changes the content of $x$ from $i$ to $j'$.
In $\mathrm{rw}(\mathbf{T})$ this implies moving the bold $i$ from the unprimed reading word to the left through 
elements $i$ and $j$ corresponding to rows $r_x, r_x +1,\ldots, r_z$, then through elements $i$ and $j$ in the primed 
reading word corresponding to columns $c_z-1, \ldots, c_x$, and then changing that $i$ to $j$ which corresponds to 
cell $x$. But according to Lemma~\ref{lemma.chains}, the letters $i$ and $j$ in these rows and columns are all bracketed 
with each other, since for every $j$ or $j'$ in the chain there is a corresponding $i$ or $i'$ in the North-Western cell. 
(Notice that there cannot be any other letter $j$ or $j'$ outside of the chain in rows $r_x +1,\ldots, r_z$ and in columns 
$c_z-1, \ldots, c_x$.) Thus, moving the bold $i$ to the left in $\mathrm{rw}(\mathbf{T})$ does not change the bracketing 
relations. Changing it to $j$ makes it the leftmost unbracketed $j$ in $\mathrm{rw}(\mathbf{T}')$.
Therefore, the bold $j$ in $\mathrm{rw}(\mathbf{T}')$ corresponds to the primed $j$ in cell $x$ of $\mathbf{T}'$, 
and the cell $y$ of the $e_i$ crystal operator rules is thus cell $x^*$ in $(\mathbf{T}')^*$.

\smallskip

\noindent
\textbf{Case 4.1.}
Suppose $\mathbf{T}$ falls under Case (2c) of the $f_i$ crystal operator rules. There is a chain starting from cell $x$ 
(in row $r_x$ and column $c_x$) and ending at cell $x_H$ (in row $r_H$ and column $c_H$) consisting of elements 
$j$ and $j'$. Applying $f_i$ to $\mathbf{T}$ changes the content of $x$ from $i$ to $j'$ and changes the content of 
$x_H$ from $j'$ to $j$. Moving $j'$ from cell $x_H$ to cell $x$ moves the corresponding bracketed $j$ in the reading 
word $\mathrm{rw}(\mathbf{T})$ to the left, and thus does not change the $\{i,j\}$ bracketing relations in 
$\mathrm{rw}(\mathbf{T}')$. On the other hand, moving the bold $i$ from cell $x$ to cell $x_H$ and then changing it to 
$j$ moves the bold $i$ in $\mathrm{rw}(\mathbf{T})$ to the right through elements $i$ and $j$ corresponding to rows 
$r_x, r_x +1,\ldots, r_H$, and then changes it to $j$. Note that according to Lemma~\ref{lemma.chains}, each $j$ in 
rows $r_x+1, r_x +2,\ldots, r_H$ has a corresponding $i$ from rows $r_x, r_x +1,\ldots, r_H - 1$ that it is
bracketed with, and vise versa. Thus, moving the bold $i$ to the position corresponding to $x_H$ does not change the 
fact that it is the rightmost unbracketed $i$ in $\mathrm{rw}(\mathbf{T})$.
Thus, the bold $j$ in $\mathrm{rw}(\mathbf{T}')$ corresponds to the unprimed $j$ in cell $x_H$ of $\mathbf{T}'$.

\smallskip

\noindent
\textbf{Case 4.2.}
Suppose $\mathbf{T}$ has a primed bold $i$ and $\mathbf{T}^*$ falls under Case (2c) of the $f_i$ crystal operator rules.
This means that there is a chain (expanding in North and East directions) in $\mathbf{T}$ starting from $i'$ in cell $x^*$
and ending in cell $x_H^*$ with content $i$ consisting of elements $i$ and $j'$. The crystal operator $f_i$ changes 
the content of cell $x^*$ from $i'$ to $i$ and changes the content of $x_H^*$ from $i$ to $j'$. For the reading word 
$\mathrm{rw}(\mathbf{T})$ this means moving the bracketed $i$ in the unprimed reading word to the right 
(which does not change the bracketing relations) and moving the bold $i$ in the primed reading word through letters
$i$ and $j$ corresponding to columns $c_x, c_x +1 ,\ldots, c_H$, which are bracketed with each other according to 
Lemma~\ref{lemma.chains}.
Thus, after changing the bold $i$ to $j$ makes it the leftmost unbracketed $j$ in $\mathrm{rw}(\mathbf{T}')$. Hence the
bold primed $j$ in $\mathbf{T}'$ corresponds to cell $x_H^*$.
Therefore $y$ from the $e_i$ crystal operator rules is cell $x_H$ of $(\mathbf{T}')^*$.
\end{proof}

%%%%%%%%%%%%%%%%%%%%%%%%%%%%%%%%%%%%%%%%%%%%%%%%%%%%%%%%%
\subsection{Proof of Theorem~\ref{theorem.main3}}
Let $\mathbf{T'} = f_i(\mathbf{T})$.

\smallskip

\noindent
\textbf{Case 1.} 
If $\mathbf{T}$ (or $\mathbf{T}^*$) falls under Case (1) of the $f_i$ crystal operator rules, then according to 
Lemma~\ref{lemma.y}, $e_i$ acts on $\mathbf{T}'$ (or on $(\mathbf{T}')^*$) by changing the content of cell 
$y_W = x$ back to $i$ and changing the content of $y = x_E$ back to $j'$. Thus, the statement of the theorem is true.

\smallskip

\noindent
\textbf{Case 2.} 
If $\mathbf{T}$ (or $\mathbf{T}^*$) falls under Case (2a) of the $f_i$ crystal operator rules, then according to 
Lemma~\ref{lemma.y}, $e_i$ acts on $\mathbf{T}'$ (or on $(\mathbf{T}')^*$) by changing the content of the cell 
$y = x$ back to $i$. Thus, the statement of the theorem is true.

\smallskip

\noindent
\textbf{Case 3.} If $\mathbf{T}$ falls under Case (2b) of the $f_i$ crystal operator rules, then according to 
Lemma~\ref{lemma.y}, $e_i$ acts on cell $y=x^*$ of $(\mathbf{T}')^*$. Note that according to Lemma~\ref{lemma.chains}, 
there is a maximal chain of letters $i$ and $j'$ in $(\mathbf{T}')^*$ starting at $y$ and ending at a diagonal cell $y_T$. 
Thus, $e_i$ changes the content of cell $y=x^*$ in $(\mathbf{T}')^*$ from $j$ to $j'$, so the content of cell $x$ in 
$\mathbf{T}'$ goes back from $j'$ to $i$. Thus, the statement of the theorem is true.

\smallskip

\noindent
\textbf{Case 4.} If $\mathbf{T}$ (or $\mathbf{T}^*$) falls under Case (2c) of the $f_i$ crystal operator rules, then according 
to Lemma~\ref{lemma.y}, $e_i$ acts on cell $y=x_H$ of $\mathbf{T}'$ (or of $(\mathbf{T}')^*$). Note that according 
to Lemma~\ref{lemma.chains}, there is a maximal (since $c(x_E) \neq j'$ and $c(x_E) \neq i$) chain of letters $i$ and $j'$
in $\mathbf{T}'$ (or $(\mathbf{T}')^*$) starting at $y$ and ending at cell $y_T = x$.
Thus, $e_i$ changes the content of cell $y=x_H$ in $(\mathbf{T}')^*$ from $j$ back to $j'$ and changes the content 
of $y_T = x$ from $j'$ back to $i$. Thus, the statement of the theorem is true.
